%%%%%%%%%%%%%%%%%%%%%%%%%%%%%%%%%%%%%%%%%%%%%%%%%%%%%%%%%%%%%%%%%%%
%                                                                 %
%                 Packages / Grundeinstellungen                   %
%                                                                 %
%%%%%%%%%%%%%%%%%%%%%%%%%%%%%%%%%%%%%%%%%%%%%%%%%%%%%%%%%%%%%%%%%%%

% Erstellen eines PDF/A 4 
\DocumentMetadata{
    pdfversion=2.0,
    pdfstandard=A-4,
}

% Festlegung des Allgemeinen Dokumentenformat
\documentclass[a4paper,12pt,parskip=half,headsepline,DIV=12,numbers=noenddot]{scrartcl}

%%%%%% Muss in die documentclass %%%%%%%%
%BCOR=12mm, Korrektur fuer die Bindung
%DIV12 DIV-Wert fuer die Erstellung des Satzspiegels

% Keine floats in andere Sections
\usepackage[section]{placeins}

% Weitere Pakete
\usepackage{microtype}
\usepackage{caption}
\usepackage{fontspec}
\usepackage{pdflscape}
\usepackage{float}
\usepackage{dirtree}
\usepackage{subcaption}
\usepackage{enumitem}

% Booktabs Tabellen
\usepackage{tabularray}
\UseTblrLibrary{booktabs}
\DefTblrTemplate{contfoot-text}{normal}{Fortsetzung auf nächster Seite}
\SetTblrTemplate{contfoot-text}{normal}
\DefTblrTemplate{conthead-text}{normal}{}
\SetTblrTemplate{conthead-text}{normal}

% Um Captions in Tabellen zu deaktivieren 
%\DefTblrTemplate{caption-tag}{default}{}
%\DefTblrTemplate{caption-sep}{default}{}
%\DefTblrTemplate{caption-text}{default}{}

% Grafiken aus PNG Dateien einbinden
\usepackage{graphicx}

% Deutsche Sonderzeichen und Silbentrennung nutzen
\usepackage[ngerman]{babel}
\usepackage{blindtext}

% Eurozeichen einbinden
\usepackage[right]{eurosym}

% Kopf- und Fußzeilen
\usepackage[headsepline,autooneside=false]{scrlayer-scrpage}
\clearpairofpagestyles

% Schriftart 
\usepackage{lmodern}
\setmainfont{TeX Gyre Termes}
\setsansfont{TeX Gyre Adventor}

% Floatende Bilder ermöglichen
\usepackage{floatflt}

% tikz
\usepackage{tikz}
\usetikzlibrary{calc,arrows,math}
\usetikzlibrary{shapes.geometric,positioning}

%% Schaltpläne nach europäischen Richtlinien
\usepackage[european]{circuitikz}
\tikzset{x=1mm,y=1mm}

\usepackage{siunitx}
\sisetup{output-decimal-marker={,},detect-all}

% Bricht lange URLs "schön" um
\usepackage[hyphens,obeyspaces,spaces]{url}

% Paket für Textfarben
\usepackage{xcolor} 
\definecolor{LightGray}{gray}{0.9}
\usepackage[pagecolor=white]{pagecolor}

% Mathematische Symbole importieren
\usepackage{amssymb}

% Paket für Zeilenabstand
\usepackage{setspace}

% Für Bildbezeichner
\usepackage{capt-of}

% Für Stichwortverzeichnis
\usepackage{makeidx}

% Für if und while 
\usepackage{etoolbox}

% Konfiguriere das Inhaltsverzeichnis
\usepackage{tocbasic}
\DeclareTOCStyleEntries[
  raggedentrytext,
  %numwidth=0pt, if numbers=noenddot is not set
  numsep=1ex,
  dynnumwidth,
]{tocline}{chapter,section}
\DeclareTOCStyleEntries[
  linefill=\TOCLineLeaderFill,
]{tocline}{section,subsection,subsubsection,paragraph,subparagraph}

\newcommand*\tocentryformat[1]{{\rmfamily#1}}
\RedeclareSectionCommands
  [tocentryformat=\tocentryformat,tocpagenumberformat=\tocentryformat]{subsection,subsubsection,paragraph,subparagraph}

\newcommand*\tocentrysectionformat[1]{{\rmfamily\bfseries#1}}
\RedeclareSectionCommands
  [tocentryformat=\tocentrysectionformat,tocpagenumberformat=\tocentrysectionformat]{section}  
  
\DeclareTOCStyleEntries[
  pagenumberbox=\hbox,
  dynnumwidth]{tocline}{chapter,section,subsection,subsubsection,paragraph,subparagraph,figure,table}

% Für schönere Listings
\usepackage[newfloat,]{minted}
\setminted{
  frame=lines,
  framesep=2mm,
  baselinestretch=1.2,
  bgcolor=LightGray,
  fontsize=\footnotesize,
  linenos,
  breaklines=true,
  breakanywhere=true,
  autogobble,
  tabsize=2
}
\setmintedinline{}

% Keine Floats bei Listings
\newenvironment{code}[2]
  {\captionsetup{type=listing}
  \providecommand{\captiontitle}{#1}
  \providecommand{\labeltitle}{#2}
  \vspace*{0.3cm}
  }
  {
  \vspace*{-0.8cm}
  \caption{\captiontitle}
  \label{\labeltitle}
  \vspace*{0.35cm}
  }
\SetupFloatingEnvironment{listing}{}

% Nummerierung inkl. Section
\usepackage{chngcntr}
\counterwithin{table}{section}
\counterwithin{figure}{section}
\counterwithin{listing}{section}

% Abkürzungsverzeichnis
\usepackage[printonlyused, smaller, withpage]{acronym}

% Erzeugt Inhaltsverzeichnis mit Querverweisen zu den Abschnitten (PDF Version)
\usepackage[bookmarksnumbered,hyperfootnotes=false,hypertexnames=false]{hyperref}
\hypersetup{
    colorlinks=true,
    linkcolor=black,
    filecolor=blue,
    citecolor = black,      
    urlcolor=blue,
}

% Darf erst hier eingebunden werden! 
\usepackage{subfiles}
\usepackage{csquotes}

% Indexerstellung
\makeindex


%%%%%%%%%%%%%%%%%%%%%%%%%%%%%%%%%%%%%%%%%%%%%%%%%%%%%%%%%%%%%%%%%%%
%                                                                 %                    
%                   Definition Zitierstil                         %
%                                                                 %
%%%%%%%%%%%%%%%%%%%%%%%%%%%%%%%%%%%%%%%%%%%%%%%%%%%%%%%%%%%%%%%%%%%

% Zitierung nach IEEE

\usepackage[
backend=biber,
style=ieee,
autocite=inline,
]{biblatex}
\addbibresource{bibtex/hauptdatei.bib}

% Zitierung nach APA

%\usepackage[
%backend=biber,
%style=apa,
%autocite=inline,
%]{biblatex}
%\addbibresource{bibtex/hauptdatei.bib}

\setcounter{biburllcpenalty}{7000}
\setcounter{biburlucpenalty}{8000}
%%%%%%%%%%%%%%%%%%%%%%%%%%%%%%%%%%%%%%%%%%%%%%%%%%%%%%%%%%%%%%%%%%%
%                                                                 %                    
%                    Definition Deckblatt                         %
%                                                                 %
%%%%%%%%%%%%%%%%%%%%%%%%%%%%%%%%%%%%%%%%%%%%%%%%%%%%%%%%%%%%%%%%%%%

% true für Bachelorarbeit / false für Hausarbeit
\newbool{bachelorarbeit}
\setbool{bachelorarbeit}{false}

% Setze Fachbereich
\newcommand{\department}{Fachbereich II \\ Management und Informationssysteme}

% Setze Studiengang
\newcommand{\studyprogram}{Wirtschaftsinformatik B.Sc.}

% Setze Modulname (bachelorarbeit muss false sein)
\newcommand{\modulname}{Qualitätsmanagement}

% Setze Dozent:in (bachelorarbeit muss false sein)
\newcommand{\auditor}{\textbf{Dozent:in:} \> Prof. Dr. Karin Vosseberg}

% Setze Gutachter:innen (bachelorarbeit muss true sein)
\newcommand{\firstauditor}{\textbf{Erstgutachter:} \> Prof. Dr. Maxi Mustermann}
\newcommand{\secondauditor}{\textbf{Zweitgutachterin:} \> Prof. Dr. Maxi Musterfrau}

% Setze Titel und Untertitel der Abreit 
\newcommand{\thetitle}{Semesteraufgabe}
\newcommand{\thesubtitle}{Entwicklung einer Hausverwaltung}

% Setze Autor:in und MatNr.
\newcommand{\theauthor}{Junior Lesage Ekane Njoh}
\newcommand{\matriculationnumber}{40128}

% Abstand zwischen Name und MatNr. (siehe Deckblatt)
\newcommand{\myspace}{1.0cm}

% Muss in src/basic_structure/deckblatt.tex einkommentiert werden! 

\newcommand{\secondauthor}{\> Steve Aguiwo II \> MatNr. 40088\\}
\newcommand{\thirdauthor}{\> Franck Majeste Dogmo Silatsa \> MatNr. 00000\\}
\newcommand{\fourthauthor}{\> Maxi Mustermensch \> MatNr. 00000\\}
\newcommand{\fifthauthor}{\> Maxi Mustermensch \> MatNr. 00000\\}

% PDF Metadaten
%\hypersetup{pdfinfo={
%Title={\thetitle},
%Author={\theauthor}
%}}

\hypersetup{pdfinfo={
Title={\thetitle},
Author={\theauthor}
}}

%%%%%%%%%%%%%%%%%%%%%%%%%%%%%%%%%%%%%%%%%%%%%%%%%%%%%%%%%%%%%%%%%%%
%                                                                 %                    
%                     Beginn des Inhalts                          %
%                                                                 %
%%%%%%%%%%%%%%%%%%%%%%%%%%%%%%%%%%%%%%%%%%%%%%%%%%%%%%%%%%%%%%%%%%%

%%%%%%%%%%%%%%%%%%%%%%%%%%%%%%%%%%%%%%%%%%%%%%%%%%%%%%%%%%%%%%%%%%%
%  Special Characters:                                            %
%                                                                 %
%             \& \% \$ \# \_ \{ \}                                %
%             \textasciitilde (~)                                 %
%             \textasciicircum (^)                                %     
%             \textbackslash (\)                                  %                    
%      \glqq Text\grqq{} für Anführungszeichen                    %
%%%%%%%%%%%%%%%%%%%%%%%%%%%%%%%%%%%%%%%%%%%%%%%%%%%%%%%%%%%%%%%%%%%

\begin{document}



% Definition Header Sections sollen in der Kopfzeile stehen; Kopfzeile mit Unterstrich
\automark[subsection]{section}
\KOMAoptions{headsepline=true}
%\ihead{Kopfzeile innen}
%\chead{Kopfzeile außen}
\ohead{\headmark}

% Definition footer \pagemark steht für Seitennummer
%\ifoot{Fußzeile innen}
%\cfoot{Fußzeile Mitte}
\ofoot{\pagemark}

% Hier werden die Trennvorschläge inkludiert
\input{src/basic_structure/trennung.tex}

% Leere Seite am Anfang
%\thispagestyle{empty} % erzeugt Seite ohne Kopf- / Fusszeile
%\mbox{}
%\newpage

% Titelseite 
%%%%%%%%%%%%%%%%%%%%%%%%%%%%%%%%%
%           Deckblatt           %
%%%%%%%%%%%%%%%%%%%%%%%%%%%%%%%%%
\setmainfont{TeX Gyre Adventor}
\thispagestyle{empty}
\begin{figure}[h!]
	\centering
	\includegraphics[width=0.6\textwidth]{src/abbildungen/logoneu.png}
\end{figure}
\begin{center}
	\large{\textbf{\department}}\\
	\large{\textbf{\studyprogram}}\\
	\vspace{1cm}
	\ifbool{bachelorarbeit}{
		\LARGE{\textbf{Bachelorarbeit}}\\
		\large{zur Erlangung des akademischen Grades \\ Bachelor of Science}\\
	}
	{
		\large{\textbf{Modul\\ \modulname}}\\
	}
	\vspace*{\fill}
	\line(1,0){450}\\
	\doublespacing
	\textbf{\Large{\thetitle}}\\
	\textbf{\large{\thesubtitle}}\\
	\line(1,0){450}\\
\end{center}
\vspace*{\fill}
\onehalfspacing
\small{
\begin{flushleft}
	\begin{tabbing}
		\textbf{Vorgelegt von:} \hspace*{0.8cm}\= \theauthor \hspace*{\myspace}\= MatNr. \matriculationnumber \\

		%%%%%%%%%%%%%%%%%%%%%%%%%%%%%%%%%%%%%%%%%%%%%%
		%                                            %
		%  Hier weitere Autor:innen einkommentieren  %
		%   Müssen qs-abgabe-2025.tex definiert sein     %
		%	                                         %
		%%%%%%%%%%%%%%%%%%%%%%%%%%%%%%%%%%%%%%%%%%%%%%		
		\secondauthor
		\thirdauthor
		%\fourthauthor
		%\fifthauthor

		\textbf{Vorgelegt am:} \> \today\\
		\ifbool{bachelorarbeit}{
			%\firstauditor\\
			%\secondauditor\\
		}
		{
			\auditor\\
		}
	\end{tabbing}
\end{flushleft}}
\setmainfont{TeX Gyre Termes}
\newpage

% Singlespacing (Zeilenabstand) (Default)
\singlespacing
\normalsize

% Abstract falls gewünscht
%\thispagestyle{empty}
%\input{abstract}
%\newpage

% Inhaltsverzeichnis anzeigen
\pagestyle{empty}
\tableofcontents
% \newpage
\pagestyle{headings}

% Header für den Inhalt 
\KOMAoptions{headsepline=true}
\ohead{\headmark}

% Input Inhalt
\section{Einleitung}\label{Einleitung}

% Vorstellung des Projekts: Entwicklung einer Hausverwaltung
% Zielsetzung: Verbesserung der Qualität durch strukturierte Anforderungsanalyse und Testkonzept
% Relevanz des Themas im Kontext des Qualitätsmanagements und Software-Testings
% Überblick über die Inhalte der Arbeit

Die Verwaltung von Gebäuden und deren Energieverbrauch stellt in der Praxis eine zentrale Herausforderung dar.  
Insbesondere in Mehrfamilienhäusern oder Wohnanlagen ist eine effiziente und übersichtliche Erfassung von Zählerständen erforderlich, um Verbrauchsdaten transparent zu machen und eine gerechte Abrechnung zu ermöglichen.  
Im Rahmen dieses Projekts entwickeln wir als Gruppe einen Prototyp für eine Hausverwaltungssoftware, die sich auf die digitale Erfassung, Verwaltung und Analyse von Zählerständen konzentriert.\par

Die Umsetzung erfolgt als webbasierte Anwendung mit einer intuitiven Benutzeroberfläche und einer zuverlässigen Datenverarbeitung.  
Unsere Hausverwaltung ermöglicht es, Gebäude, Zähler und Verbrauchsdaten zu verwalten, Zählerablesungen zu dokumentieren und historische Verbrauchswerte grafisch darzustellen.  
Dabei werden sowohl technische als auch organisatorische Aspekte berücksichtigt, um eine realitätsnahe und funktionale Lösung zu entwickeln. \par

Ein wesentlicher Bestandteil des Projekts ist das Review der Anforderungen sowie die Entwicklung eines fundierten Testkonzepts, um sicherzustellen, dass der Prototyp stabil, fehlerresistent und effizient arbeitet.  
Im Rahmen unserer Ausarbeitung dokumentieren wir die einzelnen Projektschritte detailliert und analysieren die gewonnenen Erkenntnisse.  
Unser Ziel ist es, ein praxisnahes und gut strukturiertes System zu entwerfen, das die wesentlichen Funktionen einer Hausverwaltung abbildet.  \par

Die Entwicklung des Prototyps folgt einem iterativen Ansatz.  
Zu Beginn wurden die Anforderungen überprüft und überarbeitet, um Widersprüche oder Unklarheiten zu beseitigen.  
Anschließend wurden konkrete Testfälle definiert, um die Kernfunktionen zu validieren.  
Die Tests umfassen funktionale Prüfungen, negative Tests sowie Leistungstests, um sowohl die korrekte Funktionalität als auch die Systemgrenzen zu ermitteln.  
Schließlich wurde der Prototyp entsprechend der definierten Anforderungen und Testfälle umgesetzt und evaluiert.  \par

Mit dieser Arbeit dokumentieren wir den gesamten Entwicklungsprozess, von der Anforderungsanalyse über die Testkonzeption bis hin zur Implementierung und Evaluation des Prototyps.  \par

\newpage


\section{Anforderungsanalyse}\label{sec:anforderungsanalyse}

\subsection{Review der Anforderungen}\label{subsec:review-der-anforderungen}

% Überblick über die Anforderungen aus der Datei AnforderungenHausverwaltung.pdf
% Review-Methodik (z. B. Inspektionen, Walkthroughs, Checklisten)
% Feststellung potenzieller Unklarheiten oder Mängel in den Anforderungen

Im Rahmen dieses Projekts haben wir ein \texttt{technisches Review} nach ISO 20246 durchgeführt.
Diese Methode wurde gewählt, da sie eine frühe Fehlererkennung in der Anforderungsphase ermöglicht und sich besonders für dokumentenbasierte Analysen eignet.

Das Review-Team bestand aus allen drei Projektmitgliedern, die Analyse erfolgte in zwei Schritten:
\begin{enumerate}
    \item \textbf{Individuelle Prüfung:} Jedes Teammitglied hat alleine für sich die Anforderungen unabhängig nach definierten Kriterien überprüft.
    \item \textbf{Gemeinsame Konsolidierung}: In einer Sitzung wurden die identifizierten Probleme besprochen und Verbesserungsvorschläge erarbeitet.
\end{enumerate}

Die Überprüfung erfolgte anhand folgender Kriterien:
\begin{itemize}[noitemsep, topsep=0pt, parsep=0pt, partopsep=0pt]
    \item \texttt{Vollständigkeit:} Sind alle relevanten Aspekte der Hausverwaltung abgedeckt?
    \item \texttt{Eindeutigkeit:} Sind die Anforderungen so formuliert, dass keine Missverständnisse entstehen?
    \item \texttt{Widerspruchsfreiheit:} Gibt es logische oder inhaltliche Widersprüche?
    \item \texttt{Testbarkeit der Anforderungen:} Lassen sich die Anforderungen in konkrete Testfälle überführen?
\end{itemize}
\\
Nach Überprüfung wurden alle 11 Anforderungen analysiert.
Während einige Anforderungen lediglich präzisiert wurden, waren bei anderen inhaltliche Anpassungen erforderlich, um Unklarheiten zu beseitigen und die Testbarkeit zu gewährleisten.
Von den überprüften 11 Anforderungen:
\begin{itemize}
    \item 5 konnten unverändert übernommen werden,
    \item 3 wurden konkretisiert,
    \item 3 mussten inhaltlich angepasst werden \texttt{(z. B. neue Fehlermeldungen, Validierungsregeln)}.
\end{itemize}

Die vollständige Analyse mit konkreten Verbesserungsvorschlägen ist in folgender Tabelle dokumentiert:

\footnotesize
\begin{center}
    \begin{talltblr}[caption={Identifizierte Probleme und Verbesserungsvorschläge}, label={tab:testcases}]{width=0.9\textwidth, colspec={X[1,l,m] X[3,c,m] X[5,l,m] X[7,l,m]}}
        \toprule
        \textbf{Nr} & \textbf{Anforderung} & \textbf{Problem/ Unklarheit} & \textbf{Verbesserungsvorschlag}\\ \midrule
        1 & Gebäudestruktur (1…n Gebäude, Eingänge, Wohnungen, Zähler) & keine klare Definition von „Eingang“ Ist ein Eingang ein Gebäudeteil oder eine logische Struktur?
        & Definition eines Eingangs hinzufügen (z.B.\ „Ein Eingang ist eine physische oder logische Einheit, die Zugang zu Wohnungen ermöglicht.“).\\ \cmidrule{1-4}
        2 & verschiedene Zählertypen (Strom, Gas, Wasser) & Unklar, ob weitere Typen ergänzbar sind?
        & Klarstellung, ob die Liste erweiterbar ist und wie neue Zählertypen ergänzbar. \\ \cmidrule{1-4}
        3 & Zähler-ID & keine Vorgabe zur Länge oder zum Format der ID & Die Zähler-ID muss eindeutig sein und darf nicht mehrfach vergeben werden.
        Die ID wird automatisch nach dem Schema Gebäude-Jahr-Random generiert.\\ \cmidrule{1-4}
        4 & Datenfilterung & Unklar, welche Filtermöglichkeiten existieren (Gebäude, Zeitraum)?
        & Ergänzung von Filtern nach Gebäude, Wohnung, Zeitraum und Zählertyp.  \\ \cmidrule{1-4}
        5 & Ablesewerte & Unklar, ob rückwirkende Korrekturen möglich sind.
        & Spezifikation: Ablesewerte können nur in der Zukunft oder am aktuellen Tag eingetragen werden.
        Änderungen nur durch Admins.\\ \cmidrule{1-4}
        6 & Zähler sind über ihre ID zu finden & Was passiert, wenn eine ID nicht existiert?
        & Falls eine Zähler-ID nicht existiert, erscheint die Fehlermeldung: ungültige Zählernummer.
        Bitte überprüfen Sie Ihre Eingabe  \\ \cmidrule{1-4}
        7 & Zähler sollen abgelesen werden (Eingabe von Datum und Wert) & Gibt es eine Validierung für vergangene/future Daten?
        & Klarstellung, ob das Ablesedatum nur in der Vergangenheit oder auch in der Zukunft liegen darf. \\ \cmidrule{1-4}
        8 & Zähler und Datum laufen nur vorwärts & Fehlt eine Angabe zu Testfällen (z.
        B. wie rückdatierte Werte behandelt werden) & Testfälle für Grenzwerte (min/max Werte für Datum) spezifizieren  \\ \cmidrule{1-4}
        9 & Weitere Ableseinformationen eingeben (Ablesung, Schätzung) & Müssen Nutzer einen Ablesetyp zwingend angeben oder gibt es Standardwerte?
        & Standardwert oder Pflichtfeld definieren. \\ \cmidrule{1-4}
        10 & Ableser-Informationen eingeben (Hauswart, Mieter, Energieversorger) & Können mehrere Ableser für einen Zähler existieren?
        & Klärung, ob Mehrfachzuweisungen erlaubt sind. \\ \cmidrule{1-4}
        11 & Verbrauch berechnen und anzeigen & Sind historische Verbrauchswerte abrufbar?
        & Die Verbrauchsanzeige wird nach jeder neuen Ablesung automatisch aktualisiert.
        Keine manuelle Aktualisierung ist erforderlich.
        Historische Verbrauchsdaten werden für mindestens 12 Monate gespeichert. \\ \bottomrule
    \end{talltblr}
\end{center}
\normalsize


\subsection{Verbesserung der Anforderungen}\label{subsec:verbesserung-der-anforderungen}

% Identifizierte Probleme und Verbesserungsvorschläge
% Überarbeitung der Anforderungen nach den Review-Befunden
% Nutzen klar definierter Anforderungen für spätere Testphasen

Auf Basis unseres Reviews konnten wir die Anforderungen an das Hausverwaltungsprojekt verbessern.
Dabei wurden unklare Definitionen konkretisiert, Testbarkeit verbessert und Validierungsregeln ergänzt.

Im Vergleich zu den ursprünglichen Anforderungen haben sich insbesondere die folgenden Aspekte geändert:

\begin{itemize}
    \item Definition der Zähler-ID (eindeutig, 14-stellig, festes Format)
    \item Neue Fehlerbehandlungen für ungültige ID-Eingaben
    \item Validierungsregeln für vergangene und zukünftige Ablesewerte
    \item Optimierung der Verbrauchsanzeige mit Berücksichtigung fehlender Werte
    \item Skalierbarkeit für größere Datenmengen mit 5000+ Zählern
\end{itemize}

Nach Überlegung fanden wir es gut funktionalen von nicht funktionalen Anforderungen zu trennen.
Die Trennung zwischen funktionalen und nicht-funktionalen Anforderungen ist essenziell, um eine klare Strukturierung der Systemanforderungen zu gewährleisten.\\
Unsere funktionalen Anforderungen definieren, was das System tun soll, also welche konkreten Funktionen es bereitstellt.
Sie sind direkt testbar und beschreiben die Interaktionen zwischen Nutzern und System.\\
Nicht-funktionale Anforderungen hingegen spezifizieren wie das System diese Funktionen bereitstellen soll, also Qualitätsmerkmale wie Performance, Benutzerfreundlichkeit oder Skalierbarkeit.
Durch diese Trennung wird es einfacher, sowohl die funktionale Umsetzung als auch die technischen Rahmenbedingungen des Prototyps gezielt zu überprüfen und zu optimieren.

\subsubsection{Funktionale Anforderungen}

Die folgende Tabelle enthält die funktionalen Anforderungen unseres Hausverwaltungsprototyps.
Diese Anforderungen legen fest, welche Funktionen das System bieten muss, um eine effektive Verwaltung von Gebäuden, Zählern und Verbrauchsdaten zu ermöglichen.
Dazu gehören unter anderem das Erfassen von Zählerständen, die Filterung von Daten sowie die Berechnung und Anzeige des Verbrauchs.
Jede Anforderung ist so formuliert, dass sie klar verständlich und testbar ist.

\footnotesize
\begin{center}
    \begin{longtblr}[caption={Funktionale Anforderungen}, label={neue funktionale Anforderungen}]{width=0.9\textwidth, colspec={X[1,l,m] X[5,c,m]X[5,l,m]}}
        \toprule
        \textbf{ Nr.} & \textbf{Anforderung} & \SetCell[c=1]{c} \textbf{Beschreibung}\\ \midrule

        F1 & Gebäudestruktur verwalten & Gebäude können mehrere Eingänge haben, jede Wohnung hat eine eindeutige ID.\\ \cmidrule{1-3}
        F2 & Zählertypen verwalten & Unterstützte Typen: Strom, Gas, Wasser.
        Die Liste ist erweiterbar, indem neue Typen über eine Konfigurationsdatei durch Entwickler hinzugefügt werden.\\ \cmidrule{1-3}
        F3 & Zählerverwaltung & Jeder Zähler hat eine eindeutige ID im Format \texttt{Gebäude-Jahr-Random} (14-stellig).
        Jeder Zähler gehört zu einer Wohnung und einem Zählertyp.
        Er speichert den letzten Ablesewert, das letzte Ablesedatum und die Ablesemethode.\\ \cmidrule{1-3}
        F4 & Datenfilterung & Filter nach Gebäude, Wohnung, Zählertyp und Zeitraum.\\ \cmidrule{1-3}
        F5 & Zählerablesung & Zählerwerte können nur mit aktuellem oder zukünftigen Datum erfasst werden.
        Negative Werte sind nicht zulässig.
        Falls der neue Wert kleiner als der vorherige ist, gibt es eine Fehlermeldung.
        Admins können jedoch rückwirkende Korrekturen vornehmen, falls ein Fehler festgestellt wird. \\ \cmidrule{1-3}
        F6 & Fehlermeldungen & Falls eine Zähler-ID nicht existiert, erscheint „Die eingegebene ID existiert nicht“.
        Falls eine Wohnung keiner ID zugeordnet ist, erscheint „Dieser Zähler ist keiner Wohnung zugeordnet.“\\ \cmidrule{1-3}
        F7 & Verbrauchsanzeige & Historische Verbrauchswerte sind für die letzten 12 Monate abrufbar.
        Eine grafische Darstellung ist möglich.\\ \cmidrule{1-3}
        F8 & Ableser-Informationen & Ableser können Hauswart, Mieter oder Energieversorger sein.
        Falls keine Information vorhanden ist, wird „Unbekannt“ eingetragen.\\ \cmidrule{1-3}
        F9 & Bearbeiten und Löschen von Gebäuden & Gebäude können direkt bearbeitet oder gelöscht werden.\\ \cmidrule{1-3}
        F10 & Zurück-Buttons auf allen Seiten & Verbesserte Navigation in der Anwendung.\\ \cmidrule{1-3}
        F11 & Gebäude auswählen vor Verbrauchsanzeige & Nutzer müssen erst ein Gebäude wählen, bevor Verbrauchsdaten angezeigt werden.\\ \cmidrule{1-3}
        F12 & Direkte Weiterleitung bei nur einem Gebäude & Wenn nur ein Gebäude existiert, wird die Verbrauchsanzeige sofort geladen.\\ \cmidrule{1-3}
        F13 & Unterschiedliche Speicherung für aktuelle & historische Verbrauchsdaten: \texttt{verbrauch\_aktuell\_X.png} und \texttt{verbrauch\_historie\_X\_YYYY-MM-DD.png} werden getrennt gespeichert.\\ \bottomrule

    \end{longtblr}
\end{center}
\normalsize

\newpage
\subsubsection{Nicht-funktionale Anforderungen}

Neben der funktionalen Umsetzung muss das System bestimmte nicht-funktionale Anforderungen erfüllen.
Diese betreffen Aspekte wie Systemperformance, Skalierbarkeit, Fehlerbehandlung und Benutzerfreundlichkeit.
Während funktionale Anforderungen definieren, „was“ das System tun soll, beschreiben nicht-funktionale Anforderungen, „wie gut“ es das tun muss.
Besonders wichtig sind hier Antwortzeiten der Verbrauchsanzeige, die visuelle Darstellung der Verbrauchsdaten sowie Datenschutzaspekte im Umgang mit Zählerwerten.

\footnotesize
\begin{center}
    \begin{talltblr}[caption={Nicht-Funktionale Anforderungen}, label={nicht funktionale Anforderungen}]{width=0.9\textwidth, colspec={X[1,l,m] X[5,c,m]X[5,l,m]}}
        \toprule
        \textbf{Nr.} & \textbf{Anforderung} & \SetCell[c=1]{c} \textbf{Beschreibung}\\ \midrule

        NF1 & Zeitraum für die Verbrauchsanzeige im Diagramm sichtbar & Das Diagramm zeigt den Zeitraum der Messung an (z.B. „März 2024 - Februar 2025“)und wird automatisch aktualisiert, sobald neue Verbrauchsdaten eingegeben werden. \\ \cmidrule{1-3}
        NF2 & Letzte 12 Monate immer anzeigen (auch ohne Werte) & Die Verbrauchsanzeige berücksichtigt automatisch die letzten 12 Monate.
        Fehlende Werte werden als „0“ dargestellt.\\ \cmidrule{1-3}
        NF3 & Farbliche Kennzeichnung der Zähler in der Verbrauchsanzeige & Jeder Zähler erhält eine eindeutige Farbe zur besseren Unterscheidung.\\ \cmidrule{1-3}
        NF4 & Optimierung der Antwortzeiten & Das System soll Verbrauchsdaten in unter 2 Sekunden berechnen und anzeigen.
        Die Berechnung muss auch bei einer Last von 5000 Zählern stabil bleiben.\\ \cmidrule{1-3}
        NF5 & Datenintegrität und Konsistenz & Ablesewerte dürfen nicht rückwirkend geändert werden (außer durch Admins).\\ \cmidrule{1-3}
        NF6 & Speicherung von Verbrauchsdaten gemäß Datenschutzbestimmungen & Verbrauchsdaten dürfen nur von autorisierten Nutzern eingesehen werden.\\ \cmidrule{1-3}
        NF7 & System skalierbar für große Datenmengen & Unterstützung für mindestens 100 Gebäude und 5000 Zähler.\\ \bottomrule
    \end{talltblr}
\end{center}
\normalsize
\newpage
\section{Testkonzept}\label{sec:testkonzept}

Ein strukturiertes Testkonzept ist essenziell, um die Qualität und Stabilität der entwickelten Hausverwaltungssoftware sicherzustellen.  
Da es sich um einen Prototypen handelt, fokussieren wir uns auf technische Tests zur Funktionsprüfung und verzichten auf umfassende Usability- oder Systemtests.  
Unser Ziel ist es, sicherzustellen, dass die Kernfunktionen fehlerfrei funktionieren, Daten korrekt verarbeitet werden und das System auch unter Last stabil bleibt.
Die Tests orientieren sich an etablierten Softwaretestverfahren und wurden so konzipiert, dass sie eine möglichst hohe Abdeckung der Anforderungen gewährleisten.

\subsection{Testziele und Strategie}\label{subsec:testziele-und-strategie}

Für unser Testkonzept haben wir uns klare Testziele definiert, damit wir den Fokus behalten können und nicht unnötig viele Szenarien testen müssen.
Unsere Hauptziele sind:
\begin{itemize}
	\item Sicherstellen, dass die Kernfunktionen der Software korrekt arbeiten.
	\item Überprüfung, ob Module und Komponenten korrekt zusammenarbeiten.
	\item Validierung der Fehlerbehandlung durch gezielte Eingabe ungültiger Werte.
	\item Sicherstellung der Performance unter hoher Last.
\end{itemize}

Unser Testkonzept folgt einer auf \texttt{bottom-up-ansatz} basierten Strategie, bei der zunächst einzelne Komponente geprüft und anschließend Funktions- und Performancetests durchgeführt werden:
\begin{itemize}
	\item \textbf{Zunächst Unit-Tests}: Isolierte Tests einzelner Funktionen (z.B. Validierung von Zähler-IDs oder Verbrauchsdaten).
	\item \textbf{Danach Funktionstests}: Prüfung der Geschäftslogik, u.a. Zählerverwaltung, Ablesungen und Verbrauchsberechnung.
	\item \textbf{Anschließend negative Tests}: Überprüfung der Fehlerbehandlung durch ungültige Eingaben.
	\item \textbf{Schließlich Performance-Tests}: Simulation von hoher Last, um die Skalierbarkeit zu überprüfen.
\end{itemize}

Wir wissen aus der Vorlesung, dass sich Softwaretests grundsätzlich in drei Kategorien unterteilen lassen:

\begin{itemize}
	\item \textbf{White-Box-Testing:} Interne Logik der Software wird geprüft, Code-Abdeckung ist entscheidend.
	\item \textbf{Black-Box-Testing:} Tests erfolgen ohne Kenntnis des Quellcodes, Fokus liegt auf den Ein- und Ausgaben des Systems.
	\item \textbf{Gray-Box-Testing:} Kombination aus White-Box und Black-Box, teilweise Kenntnisse über den Code werden verwendet.
\end{itemize}

Wir haben uns eine Kombination aus \textbf{Black-Box-Testing} und \textbf{Gray-Box-Testing} entschieden, um die Funktionalität aus Nutzersicht zu prüfen und gezielt Fehlerfälle im Code zu analysieren.

Unsere Entscheidung, uns auf diese Techniken zu konzentrieren, basiert auf der Tatsache, dass unser
Prototyp eine Webanwendung ist, wobei der Fokus auf der Funktionalität der Schnittstellen und Datenverarbeitung liegt.
Darauf basierend, dass viele Funktionen des Systems von numerischen Eingaben abhängig sind,
haben wir ein \textbf{Boundary-value-Testing} als zentraler Bestandteil unserer Teststrategie ausgewählt.
Besonders bei der Erfassung von Verbrauchsdaten und Ablesewerten ist es entscheidend, Grenzwerte zu identifizieren und zu validieren.

\subsection{Testarten}\label{subsec:testarten}

Um unser System effizient zu testen, haben wir vier Teststufen definiert:

\footnotesize
\begin{center}
	\begin{talltblr}[caption={Testarten}, label={Testarten}]{width=0.9\textwidth, colspec={X[3,l,m] X[5,c,m]X[5,l,m]}}\toprule
		\textbf{Testverfahren} & \textbf{Ziel} & \SetCell[c=1]{c} \textbf{Begründung} \\ \midrule
		
		Unit-Tests & Einzelne Funktionen wie Datenvalidierung, ID-Format, Speicherung von Ablesewerten & Frühes Erkennen von Fehlern in einzelnen Modulen. \\ \cmidrule{1-3}
		Funktionstests  & Überprüfung der gesamten Funktionalität wie Zählerverwaltung, Ablesungen, Filterung, Verbrauchsanzeige & Sicherstellung der korrekten Umsetzung der Anforderungen. \\ \cmidrule{1-3}
		Performance-Tests  & Simulation hoher Last durch 1000+ gleichzeitige Ablesungen & Sicherstellung, dass das System auch mit vielen Gebäuden und Zählern performant bleibt. \\ \cmidrule{1-3}
		Negative Tests  & Eingabe ungültiger Werte (z.B. leere Felder, falsche ID, negatives Datum) & Überprüfung der Fehlerbehandlung und Robustheit des Systems. \\ \bottomrule

	\end{talltblr}
\end{center}
\normalsize

\subsection{Testumgebung und Testdaten}\label{subsec:testumgebung-und-testdaten}

\subsubsection{Testumgebung}

Unsere Testumgebung haben wir versucht, so einfach wie möglich zu halten, damit wir nicht den Rahmen überspringen.
Hier sind die Kernpunkte unserer Testumgebung aufgelistet:
\begin{itemize}
	\item Der Prototyp wird in einer lokalen Entwicklungsumgebung als Flask-Anwendung entwickelt und getestet.
	\item Die Tests werden mithilfe von pytest automatisiert durchgeführt.
    \item Performance-Tests erfolgen durch Simulation hoher Anfragen über eine Flask-Testumgebung.
	\item Erstellung von Testfällen erfolgt nach den Prinzipien von Äquivalenzklassenbildung und Grenzwertanalyse.
\end{itemize}

\subsubsection{Testdaten}

Zum Testen gehören auch Testdaten zur Simulation, weil wir noch bei einem Prototyp sind, dessen Einsatz in einer produktiven Umgebung geplant ist.

\begin{itemize}
	\item Eine Testdatenbank mit Dummy-Daten wird verwendet.
	\item Persistenz der Daten erfolgt über JSON-Dateien.
	\item Für Lasttests werden 1000 simultane Ablesungen simuliert.
    \item Testfälle für Grenzwerte und ungültige Werte (z.B. negative Ablesungen) wurden vorbereitet.
\end{itemize}

\subsection{Ableitung konkreter Testfällen}\label{subsec:ableitung-konkreter-testfallen}

Die gründliche Kontrolle der Funktionen ist entscheidend für die Qualität eines Softwareprojekts. Sie bestimmt, ob das Projekt erfolgreich ist oder nicht.
Daher haben wir aus unseren Anforderungen gezielt abgeleitet, um sicherzustellen, dass der Prototyp alle Anforderungen erfüllt und fehlerfrei arbeitet.
Wir haben wie oben schon aufgelistet, vier Testarten gewählt

\newpage
\footnotesize
\begin{center}
	\begin{longtblr}[caption={Testfälle für die Hausverwaltungssoftware}, label={tab:testcases}]{width=0.9\textwidth, colspec={X[2,l,m] X[3,c,m] X[3,l,m] X[3,l,m] X[3,l,m]}}
		\toprule
        \textbf{Test-ID} & \textbf{Beschreibung} & \textbf{Eingabe} & \textbf{Erwartetes Ergebnis} & \textbf{Testtyp} \\ \midrule
        TC-F6-01 & Zähler-ID existiert nicht & `9999-99999-999999` & Fehlermeldung: „Die eingegebene ID existiert nicht.“ & Negative Test \\ \cmidrule{1-5}
        TC-F5-02 & Negativer Ablesewert & `-10` & Fehlermeldung: „Ungültiger Ablesewert“ & Negative Test \\ \cmidrule{1-5}
        TC-F3-03 & Zählerlänge & `1-2024-4567823` & Fehlermeldung: „Zähler-ID muss genau 14 Zeichen haben!“ & Negative Test \\ \cmidrule{1-5}
        TC-F5-04 & Ablesedatum rückdatiert & `2000-01-01` & Fehlermeldung: „Datum darf nicht in der Vergangenheit liegen!“ & Negative Test \\ \cmidrule{1-5}
        TC-F5-05 & Ablesewert kleiner als vorheriger Wert & Neuer Wert: `50`, alter Wert: `100` & Fehlermeldung: „Neuer Wert muss größer sein als der vorherige.“ & Negative Test \\ \cmidrule{1-5}
        TC-F3-06 & Gültige Zähler-ID über die Suchfunktion eingeben & `1-2025-54871257` & Zählerdetails werden angezeigt & Funktionstest \\ \cmidrule{1-5}
        TC-F5-07 & Korrekte Ablesung speichern & Alter Wert: 100, Neuer Wert: `250` & Wert wird korrekt gespeichert & Funktionstest \\ \cmidrule{1-5}
        TC-F5-08 & Ablesedatum in der Zukunft & Datum: `01.01.2030` & Wert wird gespeichert & Funktionstest \\ \cmidrule{1-5}
        TC-F8-09 & Standard-Ableser bei fehlender Eingabe & Ableser nicht eingetragen & Standardwert „Unbekannt“ wird gespeichert & Funktionstest \\ \cmidrule{1-5}
        TC-F7-10 & Historische Verbrauchswerte anzeigen & Es wird die Schnittstelle für Historiographer mit der Gebäude-ID \(`1'\) abgerufen & Ablesungen sollen als Liste zurückgegeben werden oder als Grafik in der Weboberfläche & Funktionstest \\ \cmidrule{1-5}
        TC-F3-11 & Suchfunktion mit Teilstring & Eingabe: `123` & Zeigt alle Zähler mit `123` in der ID & Funktionstests \\ \cmidrule{1-5}
        TC-NF4-12 & Massive Ablesungen & 10000 Ablesungen & Es werden alle Ablesungen in maximal 60 Sekunden gespeichert und keine Daten gehen verloren & Performance-Tests \\ \cmidrule{1-5}
        TC-NF4-13 & Antwortzeit-Test & Es wird die index-Seite aufgerufen & Innerhalb von wenigen Millisekunden eine Antwort geliefert & Performance-Test \\ \cmidrule{1-5}
        TC-NF7-14 & Massive Zählererstellung & 10000 Strom-Zähler & Alle Zähler werden hinzugefügt ohne zu lange Wartezeit & Performance-Test \\ \cmidrule{1-5}
        TC-NF7-15 & Massive Gebäude erstellen & 10000 Gebäude & Alle Gebäude werden hinzugefügt ohne zu lange Wartezeit & Performance-Test \\ \cmidrule{1-5}
        TC-F1-16 & Datenspeicherung und Datenabruf im JSON-Format & Dummy-Gebäude-Daten & Gebäude-Daten sollen gespeichert und abgerufen werden können.\ & Unit-Test \\ \cmidrule{1-5}
        TC-F3-17 & Zähler-ID-Generierung & 7 als Gebäude-ID und das aktuelle Jahr & Es soll eine gültige Zähler-ID generiert werden.\ & Unit-Test \\ \bottomrule
    \end{longtblr}
\end{center}
\normalsize

Unser Testkonzept ist also sehr umfassend, deckt die meisten möglichen Szenarien ab und stellt sicher, dass der Prototyp funktionsfähig, robust und performant ist.
Durch die gewählte Kombination aus unseren Testarten wollen wir eine praxisnahe Qualitätssicherung erreichen.

\newpage

\section{Prototypische Umsetzung der Hausverwaltung}\label{sec:prototypische-umsetzung-der-hausverwaltung}

\subsection{Software-Architektur und Technologien}\label{subsec:software-architektur-und-technologien}


Die prototypische Umsetzung der Hausverwaltung basiert auf einer webbasierten Client-Server-Architektur, bei der das Backend die Geschäftslogik verwaltet und das Frontend die Benutzeroberfläche bereitstellt.\par
Diese Architektur ermöglicht eine klare Trennung von Datenverarbeitung und Präsentation, wodurch die Wartbarkeit und Skalierbarkeit der Anwendung verbessert wird.

\subsubsection{Verwendete Technologien}

Für die Umsetzung des Prototyps haben wir bewusst Technologien gewählt, die eine einfache Entwicklung, Testbarkeit und Skalierbarkeit unterstützen.
Die folgende Tabelle gibt einen Überblick über die eingesetzten Technologien:

\footnotesize
\begin{center}
    \begin{talltblr}[caption={Verwendete Technologien}, label={tab:technologien}]{width=0.9\textwidth, colspec={X[3,l,m] X[7,c,m]X[5,l,m]}}
        \toprule
        \textbf{Technologie} & \textbf{Einsatzbereich} & \textbf{Begründung} \\ \midrule
        Python 3 & Backend-Logik, API & Einfache Entwicklung und große Auswahl an Bibliotheken für Web- und Datenverarbeitung \\ \cmidrule{1-3}
        Flask & Web-Framework für das Backend & Leichtgewichtiges Framework für die schnelle Entwicklung von Webanwendungen \\ \cmidrule{1-3}
        HTML, CSS, Jinja2 & Frontend und Templating & Ermöglicht dynamische Weboberflächen mit serverseitigem Rendering \\ \cmidrule{1-3}
        JSON & Datenspeicherung & Einfache persistente Speicherung von Gebäuden, Zählern und Ablesewerten \\ \cmidrule{1-3}
        Pytest & Testautomatisierung & Framework zur strukturierten Implementierung und Ausführung von Unit- und Funktionstests \\ \cmidrule{1-3}
        Matplotlib & Visualisierung von Verbrauchsdaten & Erstellung von Diagrammen zur Verbrauchsanalyse \\ \bottomrule
    \end{talltblr}
\end{center}
\normalsize

\subsubsection{Architekturübersicht}

Unsere Anwendung folgt dem Model-View-Controller (MVC)-Ansatz, um eine strukturierte Trennung zwischen Daten, Logik und Darstellung zu gewährleisten:
\begin{itemize}
    \item \textbf{Model (M):} Datenverwaltung erfolgt über JSON-Dateien, die Gebäude-, Zähler- und Ablesedaten speichern.
    \item \textbf{View (V):} Die Weboberfläche wird über HTML und Jinja2-Templates dynamisch generiert.
    \item \textbf{Controller (C):} Die Geschäftslogik ist in Flask implementiert und verarbeitet Nutzeranfragen.
\end{itemize}

Die folgende Abbildung zeigt die grundlegende Architektur unserer Hausverwaltungssoftware:

\begin{figure}[H] \centering \includegraphics[width=0.9\textwidth]{architektur.png} % Hier könnte ein Architekturdiagramm eingefügt werden 
    \caption{Architektur der Hausverwaltungssoftware} \label{fig:architektur} 
\end{figure}

Unsere Architektur ist speziell für einen Prototyp konzipiert, kann aber mit minimalem Aufwand für eine produktive Umgebung weiterentwickelt werden.

\subsection{Implementierung}\label{subsec:implementierung}

Unser Prototyp haben wir modular aufgebaut und er besteht aus den folgenden Kernkomponenten:

\footnotesize
\begin{center}
	\begin{talltblr}[caption={Kernkomponente}, label={tab:component}]{width=0.9\textwidth, colspec={X[3,l,m] X[7,c,m]}}\toprule

        \textbf{Modul} & \textbf{Beschreibung}\\ \midrule
        Flask-Backend & Verantwortlich für die Geschäftslogik der Anwendung, einschließlich der Verarbeitung von Ablesedaten und der Verwaltung der JSON-Datenbank. \\ \cmidrule{1-2}
        Web-Frontend & HTML/CSS und Jinja2 für die Benutzeroberfläche zur Verwaltung von Gebäuden, Zählern und Ablesewerten. \\ \cmidrule{1-2}
        Datenverwaltung & Speicherung und Abruf von Daten in JSON-Dateien (gebäude.json, zähler.json, ablesungen.json). \\ \cmidrule{1-2}
        Ablesungshistorie & Verwaltung und Validierung der Ablesungen sowie Berechnung von Verbrauchswerten. \\ \cmidrule{1-2}
        Diagrammerstellung & Generierung von Verbrauchsdiagrammen mit Matplotlib zur Visualisierung der historischen Daten. \\ \cmidrule{1-2}
        Testautomatisierung & Automatische Tests mit Pytest zur Validierung der wichtigsten Funktionen (Unit-Tests, Funktionstests, Performance-Tests). \\ \bottomrule
    \end{talltblr}
\end{center}
\normalsize

\subsubsection{Wichtige Funktionen und Codeausschnitte}

Unsere Geschäftslogik besteht aus mehreren Funktionen, die wir hier nicht alle auflisten können, daher haben wir uns entschieden nur die wichtigsten zu zeigen.

\textbf{Funktion:} \texttt{ ablesung\_hinzufuegen()}

Die Funktion verarbeitet die eingehende Ablesedaten, validiert sie und speichert sie in der JSON-Datenbank.
Es erfolgt zuerst eine Überprüfung, ob die erforderlichen Werte vorhanden und gültig sind.\\
Danach wird geprüft, ob der Zähler zur richtigen gebäude-ID gehört, wenn ja dann erfolgt eine Validierung, ob der neue Ablesewert nicht kleiner als der vorherige ist.
Sind alle Bedingungen erfüllt, wird die Ablesung gespeichert und eine Erfolgsmeldung zurückgegeben.


\begin{code}{code Example one}{breaking}
    \begin{minted}{python}
    @app.route("/ablesung/hinzufuegen", methods=["POST"])
        def ablesung_hinzufuegen():
        ablesungen = load_json(ABLESUNG_FILE)
        zaehler = load_json(ZAEHLER_FILE)
        gebaeude = load_json(GEBAEUDE_FILE)

        # JSON-Daten aus der Anfrage abrufen
        data = request.get_json()
        if not data:
            return jsonify({"error": "Fehlende oder ungültige JSON-Daten"}), 400

        print("Empfangene JSON-Daten:", data)

        try:
            gebaeude_id = data.get("gebaeude_id")
            zaehler_id = data.get("zaehler_id")
            datum = data.get("datum")
            wert = int(data.get("wert"))
            ableser = data.get("ableser", "Unbekannt")

            if not gebaeude_id or not zaehler_id or not datum or wert is None:
                return jsonify({"error": "Fehlende Eingaben"}), 400

            heutiges_datum = datetime.now().date()
            eingabe_datum = datetime.strptime(datum, "%Y-%m-%d").date()

            print(heutiges_datum)

            if eingabe_datum < heutiges_datum:
                return jsonify({"error": "Datum darf nicht in der Vergangenheit liegen!"}), 400


        except ValueError:
            return jsonify({"error": "Ungültiger Zahlenwert für Ablesung"}), 400

        # Prüfen, ob der gewählte Zähler wirklich zu diesem Gebäude gehört
        if not any(z["id"] == zaehler_id and str(z["gebaeude_id"]) == str(gebaeude_id) for z in zaehler):
            return jsonify({"error": "Ungueltiger Zaehler fuer dieses Gebaeude!"}), 400

        # Validierung des Ablesewerts
        if wert < 0:
            return jsonify({"error": "Ungueltiger Ablesewert"}), 400

        # Überprüfung auf vorherige Ablesewerte
        vorherige_ablesungen = [a for a in ablesungen if a["zaehler_id"] == zaehler_id]
        if vorherige_ablesungen:
            letzter_wert = max(a["wert"] for a in vorherige_ablesungen)
            if wert < letzter_wert:
                return jsonify({"error": "Neuer Ablesewert muss groesser sein als der vorherige"}), 400

        # Ablesung speichern
        neue_ablesung = {
            "gebaeude_id": gebaeude_id,
            "zaehler_id": zaehler_id,
            "datum": datum,
            "wert": wert,
            "ableser": ableser
        }
        ablesungen.append(neue_ablesung)
        save_json(ABLESUNG_FILE, ablesungen)

        return jsonify({"message": "Ablesung erfolgreich gespeichert", "ableser": ableser}), 201
    \end{minted}
\end{code}

\textbf{Funktion: }\texttt{verbrauchsanzeige()}

Diese Funktion berechnet den Verbrauch und generiert eine grafische Darstellung für den User.
Die Funktionsweise haben wir so einfach wie möglich gehalten.
Es werden zuerst die Verbrauchsdaten aus der JSON-Datei abgerufen und basierend auf der Gebäude-ID gefiltert dann wird ein Diagramm\\
mit Matplotlib zur Visualisierung des Verbrauchs erstellt.
Anschließend wird das Diagramm gespeichert und auf der Weboberfläche angezeigt.

\begin{code}{Code Example Two}{breaklines}
    \begin{minted}{python}
        @app.route("/verbrauch", methods=["GET"])
        def verbrauchsanzeige():
        ablesungen = load_json(ABLESUNG_FILE)
        gebaeude = load_json(GEBAEUDE_FILE)
        selected_gebaeude = request.args.get("gebaeude_id")

        if not selected_gebaeude:
            return render_template("verbrauch.html", gebaeude=gebaeude, selected_gebaeude=None, no_data=True)

        try:
            selected_gebaeude = int(selected_gebaeude)
        except ValueError:
            return "Fehler: Ungültige Gebäude-ID!", 400

        # Verbrauchsdaten filtern
        ablesungen = [a for a in ablesungen if str(a["gebaeude_id"]) == str(selected_gebaeude)]
        if not ablesungen:
            return render_template("verbrauch.html", gebaeude=gebaeude, selected_gebaeude=selected_gebaeude, no_data=True)

        # Diagramm generieren
        plt.figure(figsize=(10, 5))
        for zaehler_id in set(a["zaehler_id"] for a in ablesungen):
            daten = sorted([a for a in ablesungen if a["zaehler_id"] == zaehler_id], key=lambda x: x["datum"])
            x = [datetime.strptime(a["datum"], "%Y-%m-%d") for a in daten]
            y = [a["wert"] for a in daten]
            plt.plot(x, y, marker="o", linestyle="-", label=f"Zähler {zaehler_id}")

        plt.xlabel("Datum")
        plt.ylabel("Verbrauch")
        plt.legend()
        plt.grid(True)

        save_path = os.path.join("static", f"verbrauch_{selected_gebaeude}.png")
        plt.savefig(save_path)
        plt.close()

        return render_template("verbrauch.html", gebaeude=gebaeude, selected_gebaeude=selected_gebaeude, verbrauchspfad=save_path)

    \end{minted}
\end{code}

\subsubsection{Herausforderungen während der Umsetzung}

Während der entwicklung unseres Prototyps sind wir auf mehrere Herausforderungen gestoßen, die wir mit verschiedenen Lösungsansätzen bewältigt haben:

\begin{center}
	\begin{talltblr}[caption={Herausforderungen}, label={tab:problem}]{width=0.9\textwidth, colspec={X[3,l,m] X[7,c,m]}}\toprule
        \textbf{Herausforderung} & \textbf{Lösung}\\ \midrule
        Validierung der Ablesedaten & Implementierung von Prüfungen für ID-Format, Wertebereiche und Duplikate \\ \cmidrule{1-2}
        Simulation von Lasttests & Nutzung von Pytest, um 1000+ Ablesungen gleichzeitig zu simulieren \\ \cmidrule{1-2}
        Darstellung der Verbrauchsdaten & Speicherung und Abruf von Daten in JSON-Dateien (gebäude.json, zähler.json, ablesungen.json)\\ \cmidrule{1-2}
        Fehlermeldungen und UI-Feedback & Klare Fehlermeldungen und strukturierte JSON-Antworten implementiert \\ \bottomrule
    \end{talltblr}
\end{center}

Durch die Erfahrung, die wir bei der Umsetzung vom Prototyp gemacht haben, können wir bereits sagen, dass unsere JSON-basierte Datenverwaltung für diesen Fall ausreicht, jedoch für eine produktive Umgebung wäre eine Umstellung auf eine relationale Datenbank sinnvoll.\par


\subsection{Anwendung des Testkonzepts}\label{subsec:anwendung-des-testkonzepts}
% 3️⃣ \subsection{Anwendung des Testkonzepts}
% → Wie haben wir unser Testkonzept praktisch angewendet?
% → Hier können wir unsere Testfälle dokumentieren und direkt Testergebnisse analysieren.
% → Beispiel: „Die Performance-Tests wurden mit 10.000 gleichzeitigen Ablesungen durchgeführt, was die Skalierbarkeit bestätigt.“

\subsubsection{Überblick über die Testergebnisse}



%TODO: Ergebisse vervollständigen

In diesem Abschnitt werden die durchgeführten Tests dokumentiert.
Dabei wurden verschiedene Testarten angewandt, um die Funktionsweise und Stabilität des Systems zu validieren.
Die Testergebnisse basieren auf einer Kombination aus automatisierten Tests mit pytest und manuellen Überprüfungen in der Weboberfläche.

\footnotesize
\begin{center}
\begin{longtblr}[caption={Testfälle für die Hausverwaltungssoftware}, label={tab:testcases}]{width=0.9\textwidth, colspec={X[1,l,m] X[3,c,m] X[3,l,m] X[3,l,m] X[3,l,m]}}\toprule

        \textbf{Test-ID} & \textbf{Eingabe} & \textbf{Erwartetes Ergebnis} & \textbf{Tatsächliches Ergebnis} & \textbf{Status} \\ \midrule
        TC-001 & `999-9999-9999` & Fehlermeldung: „Die eingegebene ID existiert nicht.“ & Fehlermeldung wurde korrekt ausgegeben & Bestanden\\ \cmidrule{1-5}
        TC-002 & `-10` & Fehlermeldung: „Ungültiger Ablesewert“ & Fehlermeldung wurde korrekt ausgegeben & Bestanden \\ \cmidrule{1-5}
        TC-003 & `1-2024-4567823` & Fehlermeldung: „Zähler-ID muss genau 14 Zeichen haben!“ & Fehlermeldung wurde korrekt ausgegeben & Bestanden\\ \cmidrule{1-5}
        TC-004 & `2000-01-01` & Fehlermeldung: „Datum darf nicht in der Vergangenheit liegen!“ & Fehlermeldung wurde korrekt ausgegeben & Bestanden\\ \cmidrule{1-5}
        TC-005 & Neuer Wert: `50`, alter Wert: `100` & Fehlermeldung: „Neuer Wert muss größer sein als der vorherige.“ & Fehlermeldung wurde korrekt ausgegeben & Bestanden\\ \cmidrule{1-5}
        TC-006 & `1-2025-5487` & Zählerdetails werden angezeigt & Zählerdetails wurden angezeigt & Bestanden\\ \cmidrule{1-5}
        TC-007 & Alter Wert: 100, Neuer Wert: `250` & Wert wird korrekt gespeichert & Die Ablesewerte wurden gemäß den Spezifikationen korrekt gespeichert & Bestanden\\ \cmidrule{1-5}
        TC-008 & Datum: `01.01.2030` & Wert wird gespeichert & Die Ablesewerte wurden gemäß den Spezifikationen korrekt gespeichert & Bestanden\\ \cmidrule{1-5}
        TC-009 & Ableser nicht eingetragen & Standardwert „Unbekannt“ wird gespeichert & Die Ablesewerte wurden gemäß den Spezifikationen korrekt gespeichert & Bestanden\\ \cmidrule{1-5}
        TC-010 & Es wird die Schnittstelle für Verbauchshistorie mit der Gebäude-ID "1" abgerufen & Ablesungen sollen als Liste zurückgegeben werden oder als Grafik in der Weboberfläche & Ablesungen wurden als List über die API und als Grafik über die Weboberfläche zurückgegeben & Bestanden\\ \cmidrule{1-5}
        TC-011 & Eingabe: `123` & Zeigt alle Zähler mit `123` in der ID & Alle Zähler beinhaltend "123" wurden angezeigt & Bestanden \\ \cmidrule{1-5}
        TC-012 & 10000 Ablesungen & Es werden alle Ablesungen in maximal 60 Sekunden gespeichert und keine Daten gehen verloren & Es sind keine Daten verloren gegangen & Bestanden \\ \cmidrule{1-5}
        TC-013 & Es wird die index-Seite aufgerufen & Innerhalb von wenigen Millisekunden eine Antwort geliefert & Index-Seite wurde innerhalb von 0.2 Millisekunden angezeigt & Bestanden \\ \cmidrule{1-5}
        TC-014 & 10000 Strom-Zähler & Alle Zähler werden hinzugefügt ohne zu lange Wartezeit & Die erwartete Antwortzeit sollte unter 1 Sekunde bleiben, tatsächlich lag sie bei durchschnittlich 0.7 Sekunden, was innerhalb der akzeptablen Grenze liegt. Die Testdaten wurden vollständig und korrekt gespeichert & Bestanden\\ \cmidrule{1-5}
        TC-015 & 10000 Gebäude & Alle Gebäude werden hinzugefügt ohne zu lange Wartezeit & Die erwartete Antwortzeit sollte unter 1 Sekunde bleiben, tatsächlich lag sie bei durchschnittlich 0.8 Sekunden, was innerhalb der akzeptablen Grenze liegt. Die Testdaten wurden vollständig und korrekt gespeichert & Bestanden \\ \cmidrule{1-5}
        TC-016 & Dummy-Gebäude-Daten & Gebäude-Daten sollen gespeichert und abgerufen werden können. & Daten konnten korrekt gespeichert und wieder ausgelesen werden & Bestanden\\ \cmidrule{1-5}
        TC-017 & 7 als Gebäude-ID und das aktuelle Jahr & Es soll eine gültige Zähler-ID generiert werden. & ID wurde korrekt generiert & Bestanden\\ \bottomrule
    \end{longtblr}
\end{center}
\normalsize

% Definition von Testfällen basierend auf den Anforderungen
% Testfallbeschreibung mit ID, Testschritt, erwartetes Ergebnis
% Beispiele für positive und  &fälle

\subsubsection{Analyse der Testergebnisse}

Die Testergebnisse zeigen, dass der Prototyp die definierten Anforderungen weitestgehend erfüllt.
Tests haben verschiedene Erkenntnisse geliefert, die für zukünftige optimierungen genutzt werden können.
Die hohe Erfolgsquote zeigt, dass der Prototyp stabil und zuverlässig arbeitet.
Insbesondere die korrekte Verarbeitung von Zähler-IDs und Verbrauchswerten konnte nachgewiesen werden.

\textbf{Erfolgreiche Tests}

\begin{itemize}
    \item Alle Unit-Tests wurden bestanden, was zeigt, dass die Kernfunktionen (z.B.: ID-Generierung, Datenspeicherung) korrekt arbeiten.
    \item alle Funktionstests sind erfolgreich, sodass die Hausverwaltung ihre Grundfunktionen fehlerfrei ausführt.
    \item Alle negative Tests wurden auch erfolgreich ausgeführt, was zeigt, dass der Prototyp keine unzulässigen Eingabe akzeptiert.
    \item Performance-Tests bestätigen, dass das System stabil mit großen Datenmengen umgehen kann.
    IDes könnte nützlich sein, wenn der Prototyp in einer realen Umgebung für eine große Verwaltungsfirma eingesetzt werden soll.
\end{itemize}

\textbf{Verbesserungsbedarf}
Allerdings ist uns auch beim Testen einiges aufgefallen und zwar gibt es möglicherweise auch Verbesserungsbedarf, was mana an unserem Prototyp kritisieren könnte.
Da könnte man auf die folgende Punkte eingehen:
\begin{itemize}
    \item Beim Massentest von 1.000.000 gab es 7 fehlerhafte Einträge, da Zähler-IDs nicht doppelt existieren dürfen.
    Hierfür haben wir uns als Lösung ausgedacht, dass wir eine bessere ID-Prüfung vor dem Speichern einführen könnten.
    \item Die Verarbeitungsgeschwindigkeit war insgesamt gut, aber für noch größere Datenmengen könnte eine optimierte Datenbankstruktur erforderlich sein.
    Darüber hinaus könnte man mit dem Punkt Sicherheit unsere Datenpersistenz kritisieren, da die Daten ungeschützt gespeichert werden.
    Unsere Lösung hierfür wäre eine Umstellung auf eine sicherere Datenbank wie etwas MariaDb für eine produktive Umgebung.
\end{itemize}

Angesichts dieser Analyse ist festzustellen, dass unser Prototyp die wichtigsten anforderungen erfüllt und eine stabile Grundlage für eine erweiterte Version bietet.
Allerdings wäre für eine produktive Umgebung der Wechsel von JSON zu einer relationalen Datenbank sinnvoll, um eine effizientere Abfrage und bessere Skalierbarkeit zu gewährleisten.

\newpage
\section{Qualitätsmanagement-Methoden in der Softwareentwicklung}\label{sec:qualitatsmanagement-methoden-in-der-softwareentwicklung}

Aufgrund der zentralen Rolle, was Qualitätssicherung in der Softwareentwicklung spielt, sollten wir uns diese Gelegenheit nicht entgehen lassen, eine reflektierende und wissenschaftliche Betrachtung der angewendeten QS-Methoden in unserem Projekt zu liefern, daher sollten wir auch
in diesem Zusammenhang die Bedeutung von Qualitätssicherung herausarbeiten und dann konkret auf die Anwendung von QS-Methoden in unserem Projekt eingehen.


\subsection{Relevanz der Qualitätssicherung}\label{subsec:relevanz-der-qualitatssicherung}

In der Softwareentwicklung spielt die Qualitätssicherung (QS) eine entscheidende Rolle, um sicherzustellen, dass Anwendungen fehlerfrei funktionieren, effizient arbeiten und benutzerfreundlich sind \cite{DaiglGlunz2024}.

Mangelhafte QS kann zu Fehlfunktionen und Systemausfälle führen, die Datenverlust oder Nutzungseinschränkungen provozieren.
Außerdem verursachen Sicherheitslücken aufgrund von ungeprüften Schwachstellen, die von Angreifern genutzt werden könnten \cite{DaiglGlunz2024},
höheren Kosten, da nachträgliche Fehlerkorrekturen aufwendiger sind als eine frühzeitige Fehlervermeidung.
Es besteht auch das Problem der geringen Benutzerakzeptanz, da eine schlechte Usability oder Performance dazu führen kann, dass die Software nicht genutzt wird \cite{Röttgeretal2024}.

Gerade in unserem Projekt, einem Hausverwaltungssystem, ist eine präzise Qualitätssicherung unerlässlich.
Die Anwendung muss zuverlässig arbeiten, um sicherzustellen, dass Verbrauchsdaten korrekt verarbeitet und angezeigt werden, damit eine fehlerfreie Abrechnung möglich ist.
Außerdem müssen Fehlermeldung konsistenz und verständlich sein, damit Nutzer wissen, was genau schiefgelaufen ist und wie sie das Problem beheben können.
Zudem soll die Anwendung auch bei hoher Nutzerlast stabil bleiben, insbesondere wenn viele Gebäude und Zähler verwaltet werden müssen.
Um diese Risiken zu minimieren, haben wir gezielt QS-Maßnahmen in unser Projekt integriert, auf die wir im nächsten Abschnitt eingehen.

\subsection{Anwendung von QS-Methoden im Projekt}\label{subsec: anwendung-von-qs-methoden-im-projekt}

Die Qualitätssicherung unseres Projekts erfolgte in mehreren Phasen, um eine hohe Softwarequalität und Stabilität sicherzustellen. 
Dabei haben wir gezielt Maßnahmen ergriffen, die Fehler frühzeitig identifizieren und die Nachverfolgbarkeit der Anforderungen verbessern.

Unsere wichtigsten Maßnahmen zur Sicherstellung der Softwarequalität sind:

\begin{itemize}
    \item Ein Anforderung-Review nach \textbf{ISO 20246}, um Unklarheiten frühzeitig zu identifizieren und widersprüchliche Anforderungen zu vermeiden \cite{Technical-Committee}.
    \item Ein strukturiertes Testkonzept, das verschiedene Testarten wie Unit-Tests, Funktionstests, Negative Tests und Performance-Tests umfasst \cite{Röttgeretal2024}.
    \item Traceability zwischen Anforderungen und Testfällen, um sicherzustellen, dass jede relevanten Anforderungen auch tatsächlich getestet wird.
\end{itemize}

Ein zentraler Bestandteil unserer Qualitätssicherung war die direkte Verknüpfung der Anforderungen mit den Testfällen.
Dadurch konnten wir sicherstellen, dass jede implementierte Funktion einer definierten Anforderung entspricht und getestet wird.
Alle Testfälle wurden aus den überarbeiteten Anforderungen abgeleitet und unser technisches Review half uns, Anforderungen von der Implementierung zu präzisieren und spätere Fehler zu vermeiden.

Anforderungen sind ein wichtiger Punkt als Basis für die Implementierung vom Prototyp, dennoch sollten wir auch nicht vergessen, dass es noch wichtiger ist, die Prozesse zu dokumentieren, besonders die Tests. 
Eine strukturierte Testdokumentation war essenziell, um die Testergebnisse systematisch zu erfassen und nachvollziehbar zu machen.
Die Testergebnisse haben wir in einer Tabelle festgehalten, sodass jede Test-ID direkt mit einer Anforderung verknüpft ist.
Außerdem haben wir die fehlerhaften Tests analysiert, was zu gezielten Verbesserungen im Code führte.
Durch diese automatisierte Nachverfolgbarkeit konnten wir effizient Schwachstellen identifizieren und beheben, wodurch sich die Qualität des Prototyps erheblich verbesserte.

\subsection{Lessons Learned}\label{subsec:lessons-learned}

Wir können es nicht in Abrede stellen, dass uns die Anwendung von Qualitätssicherungsmethoden wertvolle Erkenntnisse gebracht hat, die über den reinen Testprozess hinausgehen.
Wie wir diese Qualitätsmaßnahmen konkret in unserem Projekt umgesetzt haben, wird im Folgenden erläutert:

\begin{itemize}
    \item \textbf{Früheres Review der Anforderungen hilft, Fehler zu vermeiden:} Unklare oder fehlerhafte Anforderungen führen zu einem erhöhten Testaufwand und Nachbesserungen.
    Durch unser strukturiertes Anforderung-Review konnten wir dies minimieren und von Anfang an präzisere Anforderungen definieren \cite{DaiglGlunz2024}.
    \item \textbf{automatisierte Tests sparen Zeit und erhöhen die Testabdeckung:} Manuelle Tests sind zeitaufwendig und fehleranfällig beispielsweise durch Konfigurationsfehler oder Eingabefehler.
    Mit \texttt{pytext} konnten wir wiederholbare Tests automatisieren und schneller fehler identifizieren \cite{Röttgeretal2024}.
    \item \textbf{Performance-Tests sind essenziell für die Skalierbarkeit:} Unsere Lasttests zeigten, dass das System stabil mit großen Datenmengen umgehen kann.
    Allerdings haben wir erkannt, dass für eine echte Skalierbarkeit eine relationale Datenbank vorteilhafter wäre als die aktuelle JSON-basierte Speicherung.
    \item \textbf{Fehlermeldungen sind genauso wichtig wie korrekte Funktionen:} Nicht nur fehlerfreie Funktionen, sondern auch verständliche Fehlermeldungen tragen zur Benutzerfreundlichkeit bei.
    Durch gezielte Tests konnten wir sicherstellen, dass Fehlermeldungen für ungültige Zähler-IDs, falsche Ablesewerte oder negative Werte klar und nachvollziehbar sind.
\end{itemize}

\newpage
\section{Fazit}\label{sec:fazit}

Im Rahmen dieses Projekts haben wir eine Hausverwaltungssoftware prototypisch entwickelt und dabei einen umfassenden softwaretechnischen Prozess durchlaufen, von der Anforderungsanalyse über die Implementierung bis hin zur Qualitätssicherung.
Unser Ziel war es, ein System zu entwerfen, dass die Erfassung, Verwaltung und Analyse von Verbrauchsdaten effizient unterstützt und dabei stabil, fehlerresistent und benutzerfreundlich ist.

Die Anforderungsanalyse bildete die Grundlage für die Entwicklung unseres Prototyps.
Durch ein technisches Review nach ISO 20246 konnten wir potenzielle Unklarheiten in den ursprünglichen Anforderungen identifizieren und gezielt verbessern.
Die Trennung zwischen funktionalen und nicht-funktionalen Anforderungen hat sich als vorteilhaft erwiesen, da sie die Strukturierung und Testbarkeit unseres Systems erleichtert hat.

Auf Basis dieser überarbeiteten Anforderungen wurde ein strukturiertes Testkonzept entwickelt.
Durch die Kombination von Unit-Tests, Funktionstests, negativen Tests und Performance-Tests konnten wir die wichtigsten Systemfunktionen validieren und sicherstellen, dass das System auch unter Last stabil bleibt.
Die durchgeführten Tests bestätigten die Korrektheit der Kernfunktionen und deckten potenzielle Optimierungsansätze auf, insbesondere in Bezug auf die Skalierbarkeit und Datenspeicherung.

Die Implementierung unseres Prototyps folgte einer modularen Architektur mit einem Flask-Backend, einer JSON-basierten Datenspeicherung und eine webbasierte Benutzeroberfläche.
Diese Architektur ermöglichte eine schnelle Entwicklung und eine einfache Testbarkeit.
Herausforderungen, wie die Validierung von Ablesedaten oder die Simulation von Lasttests, wurden durch gezielte Anpassungen und Optimierungen erfolgreich gemeistert.

Unsere Qualitätssicherungsmaßnahmen haben sich als effektiv erwiesen, um Fehler frühzeitig zu identifizieren und das System iterativ zu verbessern. 
Die Rückverfolgbarkeit zwischen Anforderungen und Testfällen trug wesentlich dazu bei, dass keine kritische Funktion ungetestet blieb.
Zudem zeigten unsere Performance-Tests, dass das System mit hohen Datenmengen umgehen kann.

Zusammenfassend hat unser Prototyp die wesentlichen Anforderungen erfolgreich umgesetzt und bietet eine solide Grundlage für eine Weiterentwicklung.
Falls das System für eine produktive Umgebung ausgebaut werden soll, wären folgende Maßnahmen sinnvoll:

\begin{itemize} 
    \item Migration von JSON zu einer relationalen Datenbank für bessere Skalierbarkeit und Sicherheit. 
    \item Erweiterung der Performance-Tests mit größeren Datenmengen und realistischen Nutzungsszenarien. 
    \item Einführung automatisierter UI-Tests zur Überprüfung der Benutzerfreundlichkeit. 
    \item Implementierung weiterer Fehlerbehandlungen und Sicherheitsmechanismen. 
\end{itemize}

Dieses Projekt hat uns gezeigt, wie entscheidend eine strukturierte Anforderungsanalyse, ein fundiertes Testkonzept und iterative Qualitätssicherungsmaßnahmen für die erfolgreiche Entwicklung einer Software sind.
Die gewonnenen Erkenntnisse und Erfahrungen aus diesem Projekt werden uns in zukünftigen Softwareentwicklungsprojekten von großem Nutzen sein.
\newpage

% Literaturverzeichnis anzeigen
\ohead{Literaturverzeichnis} % Korrektur für Header 
\phantomsection
\addcontentsline{toc}{section}{Literaturverzeichnis}
\renewcommand\refname{Literaturverzeichnis}
\printbibliography
\newpage

% Abbildungsverzeichnis anzeigen
% \ohead{\headmark}
% \listoffigures
% \addcontentsline{toc}{section}{Abbildungsverzeichnis}
% \newpage



% Listingverzeichnis anzeigen
\renewcommand{\listlistingname}{Listingverzeichnis}
\listoflistings 
\addcontentsline{toc}{section}{Listingverzeichnis}
\newpage


% Abkürzungsverzeichnis anzeigen
%\ohead{Abkürzungsverzeichnis} % Korrektur für Header 
%\section*{Abkürzungsverzeichnis}
%\input{src/basic_structure/abkuerzungen.tex}
%\addcontentsline{toc}{section}{Abkürzungsverzeichnis}
%\newpage


% Kein Header für Anhang (Deckblatt) 
\KOMAoptions{headsepline=false}
\ohead{}

% Beginn Anhang
\input{src/anhang/anhang_deckblatt.tex}

% Anhang römisch 
\renewcommand{\thesection}{\Roman{section}}
\renewcommand{\thesubsection}{\Roman{subsection}}
\setcounter{section}{0}
\counterwithin{table}{subsection}
\counterwithin{figure}{subsection}
\counterwithin{listing}{subsection}

% Header Anhang (Inhalt)
\KOMAoptions{headsepline=true}
\ohead{\headmark}
\automark{subsection}

% Input Anhang 
% \subsection{Review-Protokoll der Anforderungen an die Hausverwaltung}

Review der Anforderungen


\textbf{Methode des Reviews:}
Im Rahmen dieses Projekts haben wir ein \texttt{technisches Review} nach ISO 20246 durchgeführt.  
Diese Methode wurde gewählt, da sie eine frühe Fehlererkennung in der Anforderungsphase ermöglicht und sich besonders für dokumentenbasierte Analysen eignet. \par

Das Review-Team bestand aus allen drei Projektmitgliedern. Die Analyse erfolgte in zwei Schritten:
\begin{enumerate}
	\item \textbf{Individuelle Prüfung:} Jedes Teammitglied hat alleine für sich die Anforderungen unabhängig nach definierten Kriterien überprüft.
	\item \textbf{Gemeinsame Konsolidierung}: In einer Sitzung wurden die identifizierten Probleme besprochen und Verbesserungsvorschläge erarbeitet.
\end{enumerate}

Die Überprüfung erfolgte anhand folgender Kriterien:
\begin{itemize}[noitemsep, topsep=0pt, parsep=0pt, partopsep=0pt]
	\item \texttt{Vollständigkeit:} Sind alle relevanten Aspekte der Hausverwaltung abgedeckt?
	\item \texttt{Eindeutigkeit:} Sind die Anforderungen so formuliert, dass keine Missverständnisse entstehen?
	\item \texttt{Wiederspruchsfreiheit:} Gibt es logische oder inhaltliche Widersprüche?
	\item \texttt{Testbarkeit der Anforderungen:} Lassen sich die Anforderungen in konkrete Testfälle überführen?
\end{itemize}
	
\textbf{Ergebnisbewertung:}  
Von insgesamt \textbf{13 funktionalen Anforderungen} und \textbf{7 nicht-funktionalen Anforderungen} waren \textbf{10 Anforderungen ohne Änderungen übernehmbar}, während \textbf{5 Anforderungen} angepasst werden mussten.  
Die größten Probleme traten in folgenden Bereichen auf:
\begin{itemize}
	\item Fehlende oder unklare Definitionen (z. B. „Eingang“ im Gebäudemodell).
	\item Unklare Validierungsregeln (z. B. wie mit negativen Werten oder zukünftigen Daten umgegangen wird).
	\item Fehlende Fehlerbehandlung (z. B. wenn eine Zähler-ID nicht existiert).
\end{itemize}

Die vollständige Analyse mit konkreten Verbesserungsvorschlägen ist in folgender Tabelle dokumentiert:

\footnotesize
\begin{center}
	\begin{talltblr}[caption={Identifizierte Probleme und Verbesserungsvorschläge}, label={tab:testcases}]{width=0.9\textwidth, colspec={X[1,l,m] X[5,c,m] X[5,l,m] X[5,l,m]}}\toprule

        \textbf{Nr} & \textbf{Anforderung} & \textbf{Probelm/ Unklarheit} & \textbf{Verbesserungsvorschlag} \\ \midrule
        1 & Gebäudestruktur (1..n Gebäude, Eingänge, Wohnungen, Zähler) & Keine klare Definition von „Eingang“. Ist ein Eingang ein Gebäudeteil oder eine logische Struktur? & Definition eines Eingangs hinzufügen (z. B. „Ein Eingang ist eine physische oder logische Einheit, die Zugang zu Wohnungen ermöglicht.“).\\ \cmidrule{1-4}
        2 & Verschiedene Zählertypen (Strom, Gas, Wasser) & KUnklar, ob weitere Typen ergänzt werden können. & Klarstellung, ob die Liste erweiterbar ist und wie neue Zählertypen ergänzt werden können. \\ \cmidrule{1-4}
        3 & Zähler-ID & Keine Vorgabe zur Länge oder zum Format der ID. & Vorgabe: Die Zähler-ID besteht aus einer 14-stelligen alphanumerischen ID im Format \texttt{Gebäude-Jahr-Random}.“ \\ \cmidrule{1-4}
        4 & Datenfilterung & Unklar, welche Filtermöglichkeiten existieren (Gebäude, Zeitraum?). & Ergänzung von Filtern nach Gebäude, Wohnung, Zeitraum und Zählertyp.  \\ \cmidrule{1-4}
        5 & Ablesewerte & Unklar, ob rückwirkende Korrekturen möglich sind. & Spezifikation: Ablesewerte können nur in der Zukunft oder am aktuellen Tag eingetragen werden. Änderungen nur durch Admins.  \\ \cmidrule{1-4}
        6 & Zähler sind über ihre ID zu finden & Was passiert, wenn eine ID nicht existiert? & Definition einer Fehlermeldung für nicht gefundene IDs.  \\ \cmidrule{1-4}
        7 & Zähler sollen abgelesen werden (Eingabe von Datum und Wert) & Gibt es eine Validierung für vergangene/future Daten? & Klarstellung, ob das Ablesedatum nur in der Vergangenheit oder auch in der Zukunft liegen darf. \\ \cmidrule{1-4}
        8 & Zähler und Datum laufen nur vorwärts & Fehlt eine Angabe zu Testfällen (z. B. wie rückdatierte Werte behandelt werden) & Testfälle für Grenzwerte (min/max Werte für Datum) spezifizieren  \\ \cmidrule{1-4}
        9 & Weitere Ableseinformationen eingeben (Ablesung, Schätzung) & Müssen Nutzer einen Ablesetyp zwingend angeben oder gibt es Standardwerte? & Standardwert oder Pflichtfeld definieren. \\ \cmidrule{1-4}
        10 & Ableser-Informationen eingeben (Hauswart, Mieter, Energieversorger) & Können mehrere Ableser für einen Zähler existieren? & Klärung, ob Mehrfachzuweisungen erlaubt sind. \\ \cmidrule{1-4}
        11 & Verbrauch berechnen und anzeigen & Sind historische Verbrauchswerte abrufbar? & Definition, ob und wie Langzeitverbräuche gespeichert werden. \\ \bottomrule
    \end{talltblr}
\end{center}

\normalsize


\\
\\

\textbf{Verantwortliche Personen und Datum}

\begin{itemize}
	\item Junior Lesage Ekane Njoh
	\item Franck Majesté Silatsa Dogmo
	\item Datum: \texttt{20.02.2025}
\end{itemize}
% \newpage
\subsection{Verbesserte Anforderungen auf Review-Basis}

% Verbesserte und präzisierte Version der Anforderungen nach dem Review
% Falls Änderungen vorgenommen wurden, eine Gegenüberstellung (Vorher-Nachher-Vergleich)

Nach der Überarbeitung der ursprünglichen Anforderungen haben wir die finalen Anforderungen für die Hausverwaltung definiert.
Diese berücksichtigen die Ergebnisse des Reviews und wurden klarer formuliert, widerspruchsfrei gestaltet und um spezifische Validierungsregeln ergänzt.
Die neuen Anforderungen bilden die Basis für die Implementierung des Prototyps und stellen sicher,
dass alle relevanten Aspekte der Hausverwaltung praxisnah und technisch umsetzbar sind.

\footnotesize

\begin{center}
	\begin{talltblr}[caption={Verbesserte Anforderungen}, label={neue Anforderungen}]{width=0.9\textwidth, colspec={X[1,l,m] X[1,c,m]X[5,l,m]}}\toprule
		\textbf{Nr.} & \textbf{Anforderung} & \SetCell[c=1]{c} \textbf{Beschreibung} \\ \midrule
		
		1 & Gebäudestruktur & Ein Gebäude kann mehrere Eingänge haben. Ein Eingang ist eine physische oder logische Einheit, die Zugang zu Wohnungen ermöglicht. Jede Wohnung hat eine eindeutige ID. \\ \cmidrule{1-3}
		2  & Zählertypen & Unterstützte Typen: Strom, Gas, Wasser. Die Liste ist erweiterbar, indem neue Typen über eine Konfigurationsdatei oder Admin-Oberfläche hinzugefügt werden. \\ \cmidrule{1-3}
		3  & Zählerverwaltung & Jeder Zähler hat eine eindeutige 10-stellige ID (Format: `ZZZ-YYYY-NNNN`, z. B. `123-2024-4567`). Jeder Zähler gehört zu einer Wohnung und einem Zählertyp. Er speichert den letzten Ablesewert, das letzte Ablesedatum und die Ablesemethode. \\ \cmidrule{1-3}
		4  & Datenfilterung & Zähler können nach Gebäude, Wohnung, Zählertyp und Zeitraum gefiltert werden. Die Suche unterstützt Teilstringsuche. \\ \cmidrule{1-3}
		5  & Zählerablesung & Zählerwerte können nur mit aktuellem oder zukünftigen Datum erfasst werden. Korrekturen sind nur für Admins erlaubt. Negative Werte sind nicht zulässig. Falls der neue Wert kleiner als der vorherige ist, gibt es eine Fehlermeldung. \\ \cmidrule{1-3}
		6  & Fehlermeldungen & Falls eine Zähler-ID nicht existiert, erscheint „Die eingegebene ID existiert nicht. Bitte überprüfen Sie Ihre Eingabe.“ Falls eine Wohnung keiner ID zugeordnet ist, erscheint „Dieser Zähler ist keiner Wohnung zugeordnet.“ \\ \cmidrule{1-3}
		7  & Verbrauchsanzeige & Historische Verbrauchswerte sind für die letzten 12 Monate abrufbar. Eine grafische Darstellung ist möglich. Monatliche Verbrauchswerte können exportiert werden. \\ \cmidrule{1-3}
		8  & Ableser-Informationen & Ableser können Hauswart, Mieter oder Energieversorger sein. Falls keine Information vorhanden ist, wird „Unbekannt“ eingetragen. \\ \bottomrule

	\end{talltblr}
\end{center}

\normalsize
% \subsection{Testkonzept}

% Detaillierte Beschreibung des Testkonzepts (falls nicht vollständig im Hauptteil)
% Beschreibung der verwendeten Testmethoden
% Falls vorhanden: Teststrategie-Dokument oder Referenzen zu Teststandards (z. B. ISTQB)

\subsection{Beamer}

\LaTeX~bietet nicht nur die Möglichkeit schriftliche Ausarbeitungen in Form von Hausarbeiten zu erstellen, sondern mit Beamer eine Alternative
zu PowerPoint oder Keynote. Das Layout ist in vielen Fällen schlicht, was nicht negativ gemeint ist und erinnert im Design eher an die 
frühen 2000er. Durch einige Anpassungen ist es dennoch möglich ein schlichtes, aber funktionales Layout zu entwerfen, welches sich gut für 
Präsentationen eignet.\\
Eine solche Vorlage findet sich analog zu den Poster Vorlagen und erfordert wenig Umgewöhnung. Es können Tabellen und Abbildungen, sowie Referenzen 
wie gewohnt genutzt werden. Lediglich die Listings erfordern eine kleine Umgewöhnung, ebenso wie das Arbeiten mit der \textit{column} Umgebung.

\begin{code}{Beamer Frame}{beamer_frame}
	\begin{minted}{latex}
\section{Frame Example with table}

\begin{frame}{Frame Example with table}
	\begin{columns}

	\begin{column}{0.4\textwidth}
		This is some text in the second frame~\autocite{donovan2015go}.
		This is some text in the second~\autocite{kane2018docker}. 
	\end{column}

	\begin{column}{0.59\textwidth}
		\scriptsize{
			\begin{talltblr}[caption={\LaTeX~Sonderzeichen}]{colspec={X[l,m] X[0.5,c,m]}, rowhead=1, width=0.8\textwidth}\toprule
			Befehl                         & Ergebnis         \\ \midrule
			\textbackslash\&               & \&               \\ \cmidrule{1-2}
			\textbackslash\%               & \%               \\ \cmidrule{1-2}
			\textbackslash\$               & \$               \\ \cmidrule{1-2}
			\textbackslash\#               & \#               \\ \cmidrule{1-2}
			\textbackslash\{               & \{               \\ \cmidrule{1-2}
			\textbackslash\}               & \}               \\ \cmidrule{1-2}
			\end{talltblr}}
	\end{column}

	\end{columns} 
\end{frame}                       
	\end{minted}
\end{code}

Wie zu sehen, wird in Beamer ebenfalls mit \texttt{sections} gearbeitet. Um einen neuen Slide zu erstellen wird 
\texttt{\textbackslash begin\{frame\}} genutzt. Die \texttt{columns} Umgebung ermöglicht das Unterteilen des Frames 
in mehrere Spalten, welche daraufhin als einzelne \texttt{column} definiert werden. Dabei kann die Textbreite angegeben werden.\\
Listings müssen über eine besondere Umgebung definiert und eingebunden werden, da Beamer ansonsten nicht mit \texttt{minted} umgehen kann. 

\begin{code}{Beamer Frame mit Listing}{beamer_frame_listing}
		\inputminted{latex}{src/beamerlisting.txt}
\end{code}
% \newpage
\subsection{Konkrete Testfälle für die Hausverwaltungssoftware}

Die folgenden Testfälle überprüfen die wichtigsten Funktionen des Prototyps. Dabei werden \textbf{Unit-Tests, Integrationstests, Funktionstests und Negative Tests} berücksichtigt.

\begin{center}
	\begin{talltblr}[caption={Testfälle für die Hausverwaltungssoftware}, label={tab:testcases}]{width=0.9\textwidth, colspec={X[1,l,m] X[3,c,m] X[3,l,m] X[3,l,m] X[3,l,m]}}\toprule

        \textbf{Test-ID} & \textbf{Beschreibung} & \textbf{Eingabe} & \textbf{Erwartetes Ergebnis} & \textbf{Testtyp} \\ \midrule
        TC-001 & Zähler-ID existiert nicht & `999-9999-9999` & Fehlermeldung: *„Die eingegebene ID existiert nicht.“* & Negative Test \\ \cmidrule{1-5}
        TC-002 & Gültige Zähler-ID eingeben & `123-2024-4567` & Zählerdetails werden angezeigt & Funktionstest \\ \cmidrule{1-5}
        TC-003 & Ablesewert negativ & `-10` als Ablesewert & Fehlermeldung: *„Ungültiger Ablesewert.“* & Negative Test \\ \cmidrule{1-5}
        TC-004 & Ablesewert kleiner als vorheriger Wert & Neuer Wert: `50`, alter Wert: `100` & Fehlermeldung: *„Neuer Wert muss größer sein als der vorherige.“* & Negative Test \\ \cmidrule{1-5}
        TC-005 & Korrekte Ablesung speichern & Neuer Wert: `250` & Wert wird korrekt gespeichert & Funktionstest \\ \cmidrule{1-5}
        TC-006 & Eingabe einer zu langen Zähler-ID & `123-2024-45678` (11 Zeichen) & Fehlermeldung: *„Zähler-ID muss genau 10 Zeichen haben.“* & Negative Test \\ 
        TC-007 & Filtern nach Gebäude und Zählertyp & Gebäude: `Haus A`, Zählertyp: `Strom` & Liste zeigt nur Stromzähler von `Haus A` & Integrationstest \\ 
        TC-008 & Ablesedatum in der Zukunft & Datum: `01.01.2030` & Wert wird gespeichert & Funktionstest \\ \cmidrule{1-5}
        TC-009 & Ablesedatum rückdatiert & Datum: `01.01.2000` & Fehlermeldung: *„Datum darf nicht in der Vergangenheit liegen.“* & Negative Test \\ \cmidrule{1-5} 
        TC-010 & Standard-Ableser bei fehlender Eingabe & Ableser nicht eingetragen & Standardwert „Unbekannt“ wird gespeichert & Funktionstest \\ \cmidrule{1-5}
        TC-011 & Historische Verbrauchswerte anzeigen & Monat: `Januar` & Diagramm zeigt Verbrauchswerte für Januar & Funktionstest \\ \cmidrule{1-5}
        TC-012 & Suchfunktion mit Teilstring & Eingabe: `123` & Zeigt alle Zähler mit `123` in der ID & Integrationstest \\ \bottomrule
        \bottomrule
    \end{talltblr}
\end{center}

\newpage

% Selbstständigkeits Erklärung
\phantomsection
\addcontentsline{toc}{section}{Selbstständigkeitserklärung}

% Header für Erklärung
\ohead{Selbstständigkeitserklärung}

% Input Erklärung
\section*{Selbstständigkeitserklärung}

%\vspace{1cm}
Wir versichern, die von uns vorgelegte Arbeit selbstständig verfasst zu haben. Alle Stellen, die wörtlich oder sinngemäß aus veröffentlichten oder nicht veröffentlichten Arbeiten anderer entnommen sind, 
haben wir als entnommen kenntlich gemacht. Sämtliche Quellen und Hilfsmittel, die wir für die Arbeit benutzt haben, sind angegeben. Die Arbeit haben wir mit gleichem Inhalt bzw. in wesentlichen 
Teilen noch keiner anderen Prüfungsbehörde vorgelegt.

\vspace*{1cm}

\begin{tblr}{X[l] X[l]}
Bremerhaven, den \today & Unterschrift:\\
\end{tblr}

% Leere Abschlussseite
%\newpage
%\thispagestyle{empty} % erzeugt Seite ohne Kopf- / Fusszeile
%\mbox{}

\end{document}
