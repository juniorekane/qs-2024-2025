\newpage


\section{Anforderungsanalyse}\label{sec:anforderungsanalyse}

\subsection{Review der Anforderungen}\label{subsec:review-der-anforderungen}

% Überblick über die Anforderungen aus der Datei AnforderungenHausverwaltung.pdf
% Review-Methodik (z. B. Inspektionen, Walkthroughs, Checklisten)
% Feststellung potenzieller Unklarheiten oder Mängel in den Anforderungen

Im Rahmen dieses Projekts haben wir ein \texttt{technisches Review} nach ISO 20246 durchgeführt.
Diese Methode wurde gewählt, da sie eine frühe Fehlererkennung in der Anforderungsphase ermöglicht und sich besonders für dokumentenbasierte Analysen eignet.

Das Review-Team bestand aus allen drei Projektmitgliedern, die Analyse erfolgte in zwei Schritten:
\begin{enumerate}
    \item \textbf{Individuelle Prüfung:} Jedes Teammitglied hat alleine für sich die Anforderungen unabhängig nach definierten Kriterien überprüft.
    \item \textbf{Gemeinsame Konsolidierung}: In einer Sitzung wurden die identifizierten Probleme besprochen und Verbesserungsvorschläge erarbeitet.
\end{enumerate}

Die Überprüfung erfolgte anhand folgender Kriterien:
\begin{itemize}[noitemsep, topsep=0pt, parsep=0pt, partopsep=0pt]
    \item \texttt{Vollständigkeit:} Sind alle relevanten Aspekte der Hausverwaltung abgedeckt?
    \item \texttt{Eindeutigkeit:} Sind die Anforderungen so formuliert, dass keine Missverständnisse entstehen?
    \item \texttt{Widerspruchsfreiheit:} Gibt es logische oder inhaltliche Widersprüche?
    \item \texttt{Testbarkeit der Anforderungen:} Lassen sich die Anforderungen in konkrete Testfälle überführen?
\end{itemize}
\\
Nach Überprüfung wurden alle 11 Anforderungen analysiert.
Während einige Anforderungen lediglich präzisiert wurden, waren bei anderen inhaltliche Anpassungen erforderlich, um Unklarheiten zu beseitigen und die Testbarkeit zu gewährleisten.
Von den überprüften 11 Anforderungen:
\begin{itemize}
    \item 5 konnten unverändert übernommen werden,
    \item 3 wurden konkretisiert,
    \item 3 mussten inhaltlich angepasst werden \texttt{(z. B. neue Fehlermeldungen, Validierungsregeln)}.
\end{itemize}

Die vollständige Analyse mit konkreten Verbesserungsvorschlägen ist in folgender Tabelle dokumentiert:

\footnotesize
\begin{center}
    \begin{talltblr}[caption={Identifizierte Probleme und Verbesserungsvorschläge}, label={tab:testcases}]{width=0.9\textwidth, colspec={X[1,l,m] X[3,c,m] X[5,l,m] X[7,l,m]}}
        \toprule
        \textbf{Nr} & \textbf{Anforderung} & \textbf{Problem/ Unklarheit} & \textbf{Verbesserungsvorschlag}\\ \midrule
        1 & Gebäudestruktur (1…n Gebäude, Eingänge, Wohnungen, Zähler) & keine klare Definition von „Eingang“ Ist ein Eingang ein Gebäudeteil oder eine logische Struktur?
        & Definition eines Eingangs hinzufügen (z.B.\ „Ein Eingang ist eine physische oder logische Einheit, die Zugang zu Wohnungen ermöglicht.“).\\ \cmidrule{1-4}
        2 & verschiedene Zählertypen (Strom, Gas, Wasser) & Unklar, ob weitere Typen ergänzbar sind?
        & Klarstellung, ob die Liste erweiterbar ist und wie neue Zählertypen ergänzbar. \\ \cmidrule{1-4}
        3 & Zähler-ID & keine Vorgabe zur Länge oder zum Format der ID & Die Zähler-ID muss eindeutig sein und darf nicht mehrfach vergeben werden.
        Die ID wird automatisch nach dem Schema Gebäude-Jahr-Random generiert.\\ \cmidrule{1-4}
        4 & Datenfilterung & Unklar, welche Filtermöglichkeiten existieren (Gebäude, Zeitraum)?
        & Ergänzung von Filtern nach Gebäude, Wohnung, Zeitraum und Zählertyp.  \\ \cmidrule{1-4}
        5 & Ablesewerte & Unklar, ob rückwirkende Korrekturen möglich sind.
        & Spezifikation: Ablesewerte können nur in der Zukunft oder am aktuellen Tag eingetragen werden.
        Änderungen nur durch Admins.\\ \cmidrule{1-4}
        6 & Zähler sind über ihre ID zu finden & Was passiert, wenn eine ID nicht existiert?
        & Falls eine Zähler-ID nicht existiert, erscheint die Fehlermeldung: ungültige Zählernummer.
        Bitte überprüfen Sie Ihre Eingabe  \\ \cmidrule{1-4}
        7 & Zähler sollen abgelesen werden (Eingabe von Datum und Wert) & Gibt es eine Validierung für vergangene/future Daten?
        & Klarstellung, ob das Ablesedatum nur in der Vergangenheit oder auch in der Zukunft liegen darf. \\ \cmidrule{1-4}
        8 & Zähler und Datum laufen nur vorwärts & Fehlt eine Angabe zu Testfällen (z.
        B. wie rückdatierte Werte behandelt werden) & Testfälle für Grenzwerte (min/max Werte für Datum) spezifizieren  \\ \cmidrule{1-4}
        9 & Weitere Ableseinformationen eingeben (Ablesung, Schätzung) & Müssen Nutzer einen Ablesetyp zwingend angeben oder gibt es Standardwerte?
        & Standardwert oder Pflichtfeld definieren. \\ \cmidrule{1-4}
        10 & Ableser-Informationen eingeben (Hauswart, Mieter, Energieversorger) & Können mehrere Ableser für einen Zähler existieren?
        & Klärung, ob Mehrfachzuweisungen erlaubt sind. \\ \cmidrule{1-4}
        11 & Verbrauch berechnen und anzeigen & Sind historische Verbrauchswerte abrufbar?
        & Die Verbrauchsanzeige wird nach jeder neuen Ablesung automatisch aktualisiert.
        Keine manuelle Aktualisierung ist erforderlich.
        Historische Verbrauchsdaten werden für mindestens 12 Monate gespeichert. \\ \bottomrule
    \end{talltblr}
\end{center}
\normalsize


\textbf{Verantwortliche Personen und Datum}

\begin{itemize}
    \item Junior Lesage Ekane Njoh
    \item Franck Majesté Silatsa Dogmo
    \item Datum: \texttt{20.02.2025}
\end{itemize}

\subsection{Verbesserung der Anforderungen}\label{subsec:verbesserung-der-anforderungen}

% Identifizierte Probleme und Verbesserungsvorschläge
% Überarbeitung der Anforderungen nach den Review-Befunden
% Nutzen klar definierter Anforderungen für spätere Testphasen

Auf Basis unseres Reviews konnten wir die Anforderungen an das Hausverwaltungsprojekt verbessern.
Dabei wurden unklare Definitionen konkretisiert, Testbarkeit verbessert und Validierungsregeln ergänzt.

Im Vergleich zu den ursprünglichen Anforderungen haben sich insbesondere die folgenden Aspekte geändert:

\begin{itemize}
    \item Definition der Zähler-ID (eindeutig, 14-stellig, festes Format)
    \item Neue Fehlerbehandlungen für ungültige ID-Eingaben
    \item Validierungsregeln für vergangene und zukünftige Ablesewerte
    \item Optimierung der Verbrauchsanzeige mit Berücksichtigung fehlender Werte
    \item Skalierbarkeit für größere Datenmengen mit 5000+ Zählern
\end{itemize}

Nach Überlegung fanden wir es gut funktionalen von nicht funktionalen Anforderungen zu trennen.
Die Trennung zwischen funktionalen und nicht-funktionalen Anforderungen ist essenziell, um eine klare Strukturierung der Systemanforderungen zu gewährleisten.\\
Unsere funktionalen Anforderungen definieren, was das System tun soll, also welche konkreten Funktionen es bereitstellt.
Sie sind direkt testbar und beschreiben die Interaktionen zwischen Nutzern und System.\\
Nicht-funktionale Anforderungen hingegen spezifizieren wie das System diese Funktionen bereitstellen soll, also Qualitätsmerkmale wie Performance, Benutzerfreundlichkeit oder Skalierbarkeit.
Durch diese Trennung wird es einfacher, sowohl die funktionale Umsetzung als auch die technischen Rahmenbedingungen des Prototyps gezielt zu überprüfen und zu optimieren.

\subsubsection{Funktionale Anforderungen}

Die folgende Tabelle enthält die funktionalen Anforderungen unseres Hausverwaltungsprototyps.
Diese Anforderungen legen fest, welche Funktionen das System bieten muss, um eine effektive Verwaltung von Gebäuden, Zählern und Verbrauchsdaten zu ermöglichen.
Dazu gehören unter anderem das Erfassen von Zählerständen, die Filterung von Daten sowie die Berechnung und Anzeige des Verbrauchs.
Jede Anforderung ist so formuliert, dass sie klar verständlich und testbar ist.

\footnotesize
\begin{center}
    \begin{longtblr}[caption={Funktionale Anforderungen}, label={neue funktionale Anforderungen}]{width=0.9\textwidth, colspec={X[1,l,m] X[5,c,m]X[5,l,m]}}
        \toprule
        \textbf{ Nr.} & \textbf{Anforderung} & \SetCell[c=1]{c} \textbf{Beschreibung}\\ \midrule

        F1 & Gebäudestruktur verwalten & Gebäude können mehrere Eingänge haben, jede Wohnung hat eine eindeutige ID.\\ \cmidrule{1-3}
        F2 & Zählertypen verwalten & Unterstützte Typen: Strom, Gas, Wasser.
        Die Liste ist erweiterbar, indem neue Typen über eine Konfigurationsdatei durch Entwickler hinzugefügt werden.\\ \cmidrule{1-3}
        F3 & Zählerverwaltung & Jeder Zähler hat eine eindeutige ID im Format \texttt{Gebäude-Jahr-Random} (14-stellig).
        Jeder Zähler gehört zu einer Wohnung und einem Zählertyp.
        Er speichert den letzten Ablesewert, das letzte Ablesedatum und die Ablesemethode.\\ \cmidrule{1-3}
        F4 & Datenfilterung & Filter nach Gebäude, Wohnung, Zählertyp und Zeitraum.\\ \cmidrule{1-3}
        F5 & Zählerablesung & Zählerwerte können nur mit aktuellem oder zukünftigen Datum erfasst werden.
        Negative Werte sind nicht zulässig.
        Falls der neue Wert kleiner als der vorherige ist, gibt es eine Fehlermeldung.
        Admins können jedoch rückwirkende Korrekturen vornehmen, falls ein Fehler festgestellt wird. \\ \cmidrule{1-3}
        F6 & Fehlermeldungen & Falls eine Zähler-ID nicht existiert, erscheint „Die eingegebene ID existiert nicht“.
        Falls eine Wohnung keiner ID zugeordnet ist, erscheint „Dieser Zähler ist keiner Wohnung zugeordnet.“\\ \cmidrule{1-3}
        F7 & Verbrauchsanzeige & Historische Verbrauchswerte sind für die letzten 12 Monate abrufbar.
        Eine grafische Darstellung ist möglich.\\ \cmidrule{1-3}
        F8 & Ableser-Informationen & Ableser können Hauswart, Mieter oder Energieversorger sein.
        Falls keine Information vorhanden ist, wird „Unbekannt“ eingetragen.\\ \cmidrule{1-3}
        F9 & Bearbeiten und Löschen von Gebäuden & Gebäude können direkt bearbeitet oder gelöscht werden.\\ \cmidrule{1-3}
        F10 & Zurück-Buttons auf allen Seiten & Verbesserte Navigation in der Anwendung.\\ \cmidrule{1-3}
        F11 & Gebäude auswählen vor Verbrauchsanzeige & Nutzer müssen erst ein Gebäude wählen, bevor Verbrauchsdaten angezeigt werden.\\ \cmidrule{1-3}
        F12 & Direkte Weiterleitung bei nur einem Gebäude & Wenn nur ein Gebäude existiert, wird die Verbrauchsanzeige sofort geladen.\\ \cmidrule{1-3}
        F13 & Unterschiedliche Speicherung für aktuelle & historische Verbrauchsdaten: \texttt{verbrauch\_aktuell\_X.png} und \texttt{verbrauch\_historie\_X\_YYYY-MM-DD.png} werden getrennt gespeichert.\\ \bottomrule

    \end{longtblr}
\end{center}
\normalsize

\newpage
\subsubsection{Nicht-funktionale Anforderungen}

Neben der funktionalen Umsetzung muss das System bestimmte nicht-funktionale Anforderungen erfüllen.
Diese betreffen Aspekte wie Systemperformance, Skalierbarkeit, Fehlerbehandlung und Benutzerfreundlichkeit.
Während funktionale Anforderungen definieren, „was“ das System tun soll, beschreiben nicht-funktionale Anforderungen, „wie gut“ es das tun muss.
Besonders wichtig sind hier Antwortzeiten der Verbrauchsanzeige, die visuelle Darstellung der Verbrauchsdaten sowie Datenschutzaspekte im Umgang mit Zählerwerten.

\footnotesize
\begin{center}
    \begin{talltblr}[caption={Nicht-Funktionale Anforderungen}, label={nicht funktionale Anforderungen}]{width=0.9\textwidth, colspec={X[1,l,m] X[5,c,m]X[5,l,m]}}
        \toprule
        \textbf{Nr.} & \textbf{Anforderung} & \SetCell[c=1]{c} \textbf{Beschreibung}\\ \midrule

        NF1 & Zeitraum für die Verbrauchsanzeige im Diagramm sichtbar & Das Diagramm zeigt den Zeitraum der Messung an (z.B. „März 2024 - Februar 2025“)und wird automatisch aktualisiert, sobald neue Verbrauchsdaten eingegeben werden. \\ \cmidrule{1-3}
        NF2 & Letzte 12 Monate immer anzeigen (auch ohne Werte) & Die Verbrauchsanzeige berücksichtigt automatisch die letzten 12 Monate.
        Fehlende Werte werden als „0“ dargestellt.\\ \cmidrule{1-3}
        NF3 & Farbliche Kennzeichnung der Zähler in der Verbrauchsanzeige & Jeder Zähler erhält eine eindeutige Farbe zur besseren Unterscheidung.\\ \cmidrule{1-3}
        NF4 & Optimierung der Antwortzeiten & Das System soll Verbrauchsdaten in unter 2 Sekunden berechnen und anzeigen.
        Die Berechnung muss auch bei einer Last von 5000 Zählern stabil bleiben.\\ \cmidrule{1-3}
        NF5 & Datenintegrität und Konsistenz & Ablesewerte dürfen nicht rückwirkend geändert werden (außer durch Admins).\\ \cmidrule{1-3}
        NF6 & Speicherung von Verbrauchsdaten gemäß Datenschutzbestimmungen & Verbrauchsdaten dürfen nur von autorisierten Nutzern eingesehen werden.\\ \cmidrule{1-3}
        NF7 & System skalierbar für große Datenmengen & Unterstützung für mindestens 100 Gebäude und 5000 Zähler.\\ \bottomrule
    \end{talltblr}
\end{center}
\normalsize