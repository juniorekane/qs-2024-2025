\subsection{Prototyp-Dokumentation}

% Architektur-Diagramme (z. B. UML-Diagramme: Klassen-, Sequenz- oder Komponenten-Diagramme)
% Screenshots der Anwendung
% Falls vorhanden: Quellcode-Ausschnitte mit Erklärungen
\subsubsection{Überblick}

Dieser Prototyp wurde im Rahmen der Semesteraufgabe im Fach Qualitätsmanagement an der Hochschule Bremerhaven entwickelt. Ziel des Projekts ist die Umsetzung einer Hausverwaltung mit folgenden Kernfunktionen:

\begin{itemize}
    \item Verwaltung von Gebäuden, Zählern und Ablesewerten
    \item Verbrauchsanzeige mit historischen Daten (12 Monate)
    \item Speicherung und Validierung von Ablesedaten
    \item Visuelle Darstellung des Verbrauchs mit separaten Diagrammen für aktuelle und historische Werte
\end{itemize}

Das Projekt wurde als \textbf{Web-Anwendung mit Flask (Python)} umgesetzt. Als Datenspeicher werden \textbf{JSON-Dateien} genutzt, da der Fokus auf einem Prototyp und nicht auf einer produktiven Umgebung liegt.

\subsubsection{Architektur und Technologien}

Die Architektur folgt einem klassischen **MVC-Modell (Model-View-Controller)** mit folgenden Komponenten:

\begin{itemize}
    \item \textbf{Flask (Backend)} – Verarbeitung von Anfragen, Laden und Speichern der JSON-Daten
    \item \textbf{HTML + Jinja (Frontend)} – Dynamische Darstellung der Gebäude, Zähler und Verbrauchswerte
    \item \textbf{Matplotlib} – Visualisierung des Verbrauchs in Diagrammen
    \item \textbf{JavaScript} – Interaktive Formulare und automatische Aktualisierung der Gebäudeauswahl
\end{itemize}

\subsubsection{UML-Klassendiagramm}

\begin{figure}[h]
    \centering
    \includegraphics[width=0.7\textwidth]{uml_klassendiagramm.png}
    \caption{UML-Klassendiagramm der wichtigsten Entitäten}
\end{figure}

Das Klassendiagramm zeigt die Hauptklassen:
\begin{itemize}
    \item \texttt{Gebäude} – Beinhaltet Name, Adresse und Eingänge
    \item \texttt{Zähler} – Gehört zu einem Gebäude, speichert Typ und ID
    \item \texttt{Ablesung} – Speichert Werte mit Datum für einen bestimmten Zähler
\end{itemize}

\subsubsection{Sequenzdiagramm: Ablesung hinzufügen}

\begin{figure}[h]
    \centering
    \includegraphics[width=0.7\textwidth]{uml_sequenzdiagramm.png}
    \caption{UML-Sequenzdiagramm für das Hinzufügen einer Ablesung}
\end{figure}

\textbf{Ablauf:}
\begin{enumerate}
    \item Der Nutzer gibt eine Zähler-ID, ein Datum und einen Verbrauchswert ein.
    \item Das System überprüft die Zähler-ID und validiert die Eingabe.
    \item Die Ablesung wird gespeichert, wenn sie gültig ist.
    \item Die Verbrauchsanzeige wird aktualisiert.
\end{enumerate}

\subsubsection{Code-Struktur und wichtige Komponenten}

\textbf{Flask Backend (Python)}

\begin{code}{Flask Backend (Python)}{Verbrauchsanzeige}
    \begin{minted}{python}
        @app.route("/zaehler/hinzufuegen", methods=["POST"])
        def zaehler_hinzufuegen():
            zaehler = load_json(ZAEHLER_FILE)
            neue_id = f"{request.form['gebaeude_id']}-{datetime.now().year}-{random.randint(1000,9999)}"
            neuer_zaehler = {"id": neue_id, "gebaeude_id": request.form["gebaeude_id"], "typ": request.form["typ"]}
            zaehler.append(neuer_zaehler)
            save_json(ZAEHLER_FILE, zaehler)
            return redirect("/zaehler")
    \end{minted}
\end{code}



\begin{code}{HTML-Frontend mit Jinja}{Frontend}
    \begin{minted}{html}
        <form method="POST" action="/ablesung/hinzufuegen">
            <label>Zähler-ID:</label>
            <input type="text" name="zaehler_id" required>
            <label>Datum:</label>
            <input type="date" name="datum" required>
            <label>Wert:</label>
            <input type="number" name="wert" required>
            <button type="submit">Ablesung speichern</button>
        </form>
    \end{minted}
\end{code}

\begin{code}{Matplotlib Diagramm-Erstellung}{Bildgenerierung}
    \begin{minted}{python}
        plt.figure(figsize=(10, 5))
        for zaehler_id in set(a["zaehler_id"] for a in ablesungen):
            daten = sorted([a for a in ablesungen if a["zaehler_id"] == zaehler_id], key=lambda x: x["datum"])
            x = [datetime.strptime(a["datum"], "%Y-%m-%d") for a in daten]
            y = [a["wert"] for a in daten]
            plt.plot(x, y, marker="o", linestyle="-", label=f"Zähler {zaehler_id}")
        plt.legend()
        plt.savefig("static/verbrauchsdiagramm.png")
    \end{minted}
\end{code}

\subsubsection{Zusammenfassung und Fazit}

Der Prototyp bietet eine funktionale Umsetzung der Anforderungen mit einer benutzerfreundlichen Verwaltung von Gebäuden, Zählern und Verbrauchswerten. Die Nutzung von Flask und JSON für die Datenhaltung macht ihn leicht erweiterbar.

\textbf{Hauptmerkmale:}
\begin{itemize}
    \item Dynamische Verwaltung von Gebäuden und Zählern
    \item Historische Verbrauchsanzeige mit separaten Diagrammen
    \item Farblich unterscheidbare Zähler-Darstellung
    \item Integration der letzten 12 Monate in die Diagramme
\end{itemize}

Mögliche Erweiterungen wären eine Umstellung auf eine Datenbank sowie eine Nutzerverwaltung für Admin- und Standardbenutzerrechte.