\subsection{Prototyp-Dokumentation}

Passend zu diesem Template existieren noch zwei weitere Vorlagen für Poster. Diese unterscheiden
sich lediglich durch die Ausrichtung der Seite und sind ansonsten identisch. Der Aufbau der
Dateien und Projektstruktur ist weitestgehend identisch zu dieser Vorlage, es gilt lediglich
auf einige kleine Besonderheiten zu verweisen.\\
In der sog. \texttt{multicols} Umgebung ist es nicht möglich, mit Floatumgebungen zu arbeiten,
dies ist für fast alle Beispiele aus dieser Anleitung irrelevant, lediglich Grafiken müssen
anders eingebunden werden. Dazu finden sich jedoch Beispiele in den Vorlagen. Ebenso ist ein
Teil der Präambel in eine separate Datei ausgelagert, in dieser sollten auch nicht ohne weiteres
Änderungen vorgenommen werden. Wie in diesem Template, müssen innerhalb der Präambel Zitierstil,
sowie Autor:innen und Titel gesetzt werden.\\
Die Vorlagen können auf meiner Webseite sowohl angeschaut, als auch heruntergeladen werden, alle
Hinweise zum Kompilieren gelten für die Poster ebenso. Ergebnis dieser Vorlage ist z. B. folgendes
Poster:

% Architektur-Diagramme (z. B. UML-Diagramme: Klassen-, Sequenz- oder Komponenten-Diagramme)
% Screenshots der Anwendung
% Falls vorhanden: Quellcode-Ausschnitte mit Erklärungen