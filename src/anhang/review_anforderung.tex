\subsection{Review-Protokoll der Anforderungen an die Hausverwaltung}

Review der Anforderungen


\textbf{Methode des Reviews:}
Im Rahmen dieses Projekts haben wir ein \texttt{technisches Review} nach ISO 20246 durchgeführt.  
Diese Methode wurde gewählt, da sie eine frühe Fehlererkennung in der Anforderungsphase ermöglicht und sich besonders für dokumentenbasierte Analysen eignet. \par

Das Review-Team bestand aus allen drei Projektmitgliedern. Die Analyse erfolgte in zwei Schritten:
\begin{enumerate}
	\item \textbf{Individuelle Prüfung:} Jedes Teammitglied hat alleine für sich die Anforderungen unabhängig nach definierten Kriterien überprüft.
	\item \textbf{Gemeinsame Konsolidierung}: In einer Sitzung wurden die identifizierten Probleme besprochen und Verbesserungsvorschläge erarbeitet.
\end{enumerate}

Die Überprüfung erfolgte anhand folgender Kriterien:
\begin{itemize}[noitemsep, topsep=0pt, parsep=0pt, partopsep=0pt]
	\item \texttt{Vollständigkeit:} Sind alle relevanten Aspekte der Hausverwaltung abgedeckt?
	\item \texttt{Eindeutigkeit:} Sind die Anforderungen so formuliert, dass keine Missverständnisse entstehen?
	\item \texttt{Wiederspruchsfreiheit:} Gibt es logische oder inhaltliche Widersprüche?
	\item \texttt{Testbarkeit der Anforderungen:} Lassen sich die Anforderungen in konkrete Testfälle überführen?
\end{itemize}
	
\textbf{Ergebnisbewertung:}  
Von insgesamt \textbf{13 funktionalen Anforderungen} und \textbf{7 nicht-funktionalen Anforderungen} waren \textbf{10 Anforderungen ohne Änderungen übernehmbar}, während \textbf{5 Anforderungen} angepasst werden mussten.  
Die größten Probleme traten in folgenden Bereichen auf:
\begin{itemize}
	\item Fehlende oder unklare Definitionen (z. B. „Eingang“ im Gebäudemodell).
	\item Unklare Validierungsregeln (z. B. wie mit negativen Werten oder zukünftigen Daten umgegangen wird).
	\item Fehlende Fehlerbehandlung (z. B. wenn eine Zähler-ID nicht existiert).
\end{itemize}

Die vollständige Analyse mit konkreten Verbesserungsvorschlägen ist in folgender Tabelle dokumentiert:

\footnotesize
\begin{center}
	\begin{talltblr}[caption={Identifizierte Probleme und Verbesserungsvorschläge}, label={tab:testcases}]{width=0.9\textwidth, colspec={X[1,l,m] X[5,c,m] X[5,l,m] X[5,l,m]}}\toprule

        \textbf{Nr} & \textbf{Anforderung} & \textbf{Probelm/ Unklarheit} & \textbf{Verbesserungsvorschlag} \\ \midrule
        1 & Gebäudestruktur (1..n Gebäude, Eingänge, Wohnungen, Zähler) & Keine klare Definition von „Eingang“. Ist ein Eingang ein Gebäudeteil oder eine logische Struktur? & Definition eines Eingangs hinzufügen (z. B. „Ein Eingang ist eine physische oder logische Einheit, die Zugang zu Wohnungen ermöglicht.“).\\ \cmidrule{1-4}
        2 & Verschiedene Zählertypen (Strom, Gas, Wasser) & KUnklar, ob weitere Typen ergänzt werden können. & Klarstellung, ob die Liste erweiterbar ist und wie neue Zählertypen ergänzt werden können. \\ \cmidrule{1-4}
        3 & Zähler-ID & Keine Vorgabe zur Länge oder zum Format der ID. & Vorgabe: Die Zähler-ID besteht aus einer 14-stelligen alphanumerischen ID im Format \texttt{Gebäude-Jahr-Random}.“ \\ \cmidrule{1-4}
        4 & Datenfilterung & Unklar, welche Filtermöglichkeiten existieren (Gebäude, Zeitraum?). & Ergänzung von Filtern nach Gebäude, Wohnung, Zeitraum und Zählertyp.  \\ \cmidrule{1-4}
        5 & Ablesewerte & Unklar, ob rückwirkende Korrekturen möglich sind. & Spezifikation: Ablesewerte können nur in der Zukunft oder am aktuellen Tag eingetragen werden. Änderungen nur durch Admins.  \\ \cmidrule{1-4}
        6 & Zähler sind über ihre ID zu finden & Was passiert, wenn eine ID nicht existiert? & Definition einer Fehlermeldung für nicht gefundene IDs.  \\ \cmidrule{1-4}
        7 & Zähler sollen abgelesen werden (Eingabe von Datum und Wert) & Gibt es eine Validierung für vergangene/future Daten? & Klarstellung, ob das Ablesedatum nur in der Vergangenheit oder auch in der Zukunft liegen darf. \\ \cmidrule{1-4}
        8 & Zähler und Datum laufen nur vorwärts & Fehlt eine Angabe zu Testfällen (z. B. wie rückdatierte Werte behandelt werden) & Testfälle für Grenzwerte (min/max Werte für Datum) spezifizieren  \\ \cmidrule{1-4}
        9 & Weitere Ableseinformationen eingeben (Ablesung, Schätzung) & Müssen Nutzer einen Ablesetyp zwingend angeben oder gibt es Standardwerte? & Standardwert oder Pflichtfeld definieren. \\ \cmidrule{1-4}
        10 & Ableser-Informationen eingeben (Hauswart, Mieter, Energieversorger) & Können mehrere Ableser für einen Zähler existieren? & Klärung, ob Mehrfachzuweisungen erlaubt sind. \\ \cmidrule{1-4}
        11 & Verbrauch berechnen und anzeigen & Sind historische Verbrauchswerte abrufbar? & Definition, ob und wie Langzeitverbräuche gespeichert werden. \\ \bottomrule
    \end{talltblr}
\end{center}

\normalsize


\\
\\

\textbf{Verantwortliche Personen und Datum}

\begin{itemize}
	\item Junior Lesage Ekane Njoh
	\item Franck Majesté Silatsa Dogmo
	\item Datum: \texttt{20.02.2025}
\end{itemize}