\subsection{Review-Protokoll der Anforderungen an die Hausverwaltung}

Review der Anforderungen

% Kopie der originalen Anforderungen (AnforderungenHausverwaltung.pdf oder eine Zusammenfassung)
% Review-Protokoll:
% Methode des Reviews (z. B. Inspektion, Walkthrough)
% Liste der identifizierten Probleme/Unklarheiten
% Verbesserungsvorschläge
% Verantwortliche Personen und Datum des Reviews

\textbf{Methode des Reviews:}
\begin{itemize}[noitemsep, topsep=0pt, parsep=0pt, partopsep=0pt]
	\item Es wurde ein \texttt{technisches Review} nach ISO 20246 durchgeführt.
	\item Die Überprüfung erfolgte anhand folgender Kriterien:
	\begin{itemize}[noitemsep, topsep=0pt, parsep=0pt, partopsep=0pt]
		\item \texttt{Vollständigkeit}
		\item \texttt{Eindeutigkeit}
		\item \texttt{Wiederspruchsfreiheit}
		\item \texttt{Testbarkeit der Anforderungen}
	\end{itemize}
	\item Zusätzlich wurden relevante Inhalte aus den Vorlesungsfolien zum Thema Qualitätsmanagement, Softwaretest und Anforderungsanalyse berücksichtigt.
\end{itemize}

\textbf{Identifizierte Probleme und Unklarheiten}

\newcommand{\anforderung}[3]{%
    \subsection*{Anforderung #1: #2}
    \textbf{Problem/Unklarheit:} #3
    
    \vspace{0.3cm}
    \textbf{Verbesserungsvorschlag:}
}

\anforderung{1}{Gebäudestruktur (1..n Gebäude, Eingänge, Wohnungen, Zähler)}
{Keine klare Definition von „Eingang“. Ist ein Eingang ein Gebäudeteil oder eine logische Struktur?}
{Definition eines Eingangs hinzufügen (z. B. „Ein Eingang ist eine physische oder logische Einheit, die Zugang zu Wohnungen ermöglicht.“).}

\anforderung{2}{Verschiedene Zählertypen (Strom, Gas, Wasser)}
{Sind weitere Zählertypen möglich? Falls ja, wie werden sie erfasst?}
{Klarstellung, ob die Liste erweiterbar ist und wie neue Zählertypen ergänzt werden können.}

\anforderung{3}{Zähler eindeutig identifizierbar (Zählernummer)}
{Keine Angabe, welches Format oder welche Länge die Zählernummer haben muss.}
{Definition des Formats der Zählernummer (z. B. „Die Zählernummer besteht aus einer eindeutigen 10-stelligen alphanumerischen ID.“).}

\anforderung{4}{Auswahl von Daten per Selektion in der Struktur}
{Welche Filter- und Suchmöglichkeiten gibt es?}
{Ergänzung der Anforderungen zur Filterung (z. B. Suche nach Gebäude, Wohnung oder Zählertyp).}

\anforderung{5}{Zähler haben einen Ablesewert (ganze Zahl)}
{Was passiert bei fehlerhafter Eingabe? Kann der Wert korrigiert werden?}
{Spezifikation einer Fehlerbehandlung für falsche Eingaben.}

\anforderung{6}{Zähler sind über ihre ID zu finden}
{Was passiert, wenn eine ID nicht existiert?}
{Definition einer Fehlermeldung für nicht gefundene IDs.}

\anforderung{7}{Zähler sollen abgelesen werden (Eingabe von Datum und Wert)}
{Gibt es eine Validierung für vergangene/future Daten?}
{Klarstellung, ob das Ablesedatum nur in der Vergangenheit oder auch in der Zukunft liegen darf.}

\anforderung{8}{Zähler und Datum laufen nur vorwärts}
{Fehlt eine Angabe zu Testfällen (z. B. wie rückdatierte Werte behandelt werden).}
{Testfälle für Grenzwerte (min/max Werte für Datum) spezifizieren.}

\anforderung{9}{Weitere Ableseinformationen eingeben (Ablesung, Schätzung)}
{Müssen Nutzer einen Ablesetyp zwingend angeben oder gibt es Standardwerte?}
{Standardwert oder Pflichtfeld definieren.}

\anforderung{10}{Ableser-Informationen eingeben (Hauswart, Mieter, Energieversorger)}
{Können mehrere Ableser für einen Zähler existieren?}
{Klärung, ob Mehrfachzuweisungen erlaubt sind.}

\anforderung{11}{Verbrauch berechnen und Anzeigen}
{Sind historische Verbrauchswerte abrufbar?}
{Definition, ob und wie Langzeitverbräuche gespeichert werden.}
\\
\\

\textbf{Verantwortliche Personen und Datum}

\begin{itemize}
	\item Junior Lesage Ekane Njoh
	\item Franck Majesté Silatsa Dogmo
	\item Datum: \texttt{20.02.2025}
\end{itemize}