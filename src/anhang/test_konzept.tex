\subsection{Testkonzept}

% Detaillierte Beschreibung des Testkonzepts (falls nicht vollständig im Hauptteil)
% Beschreibung der verwendeten Testmethoden
% Falls vorhanden: Teststrategie-Dokument oder Referenzen zu Teststandards (z. B. ISTQB)

\subsection{Beamer}

\LaTeX~bietet nicht nur die Möglichkeit schriftliche Ausarbeitungen in Form von Hausarbeiten zu erstellen, sondern mit Beamer eine Alternative
zu PowerPoint oder Keynote. Das Layout ist in vielen Fällen schlicht, was nicht negativ gemeint ist und erinnert im Design eher an die 
frühen 2000er. Durch einige Anpassungen ist es dennoch möglich ein schlichtes, aber funktionales Layout zu entwerfen, welches sich gut für 
Präsentationen eignet.\\
Eine solche Vorlage findet sich analog zu den Poster Vorlagen und erfordert wenig Umgewöhnung. Es können Tabellen und Abbildungen, sowie Referenzen 
wie gewohnt genutzt werden. Lediglich die Listings erfordern eine kleine Umgewöhnung, ebenso wie das Arbeiten mit der \textit{column} Umgebung.

\begin{code}{Beamer Frame}{beamer_frame}
	\begin{minted}{latex}
\section{Frame Example with table}

\begin{frame}{Frame Example with table}
	\begin{columns}

	\begin{column}{0.4\textwidth}
		This is some text in the second frame~\autocite{donovan2015go}.
		This is some text in the second~\autocite{kane2018docker}. 
	\end{column}

	\begin{column}{0.59\textwidth}
		\scriptsize{
			\begin{talltblr}[caption={\LaTeX~Sonderzeichen}]{colspec={X[l,m] X[0.5,c,m]}, rowhead=1, width=0.8\textwidth}\toprule
			Befehl                         & Ergebnis         \\ \midrule
			\textbackslash\&               & \&               \\ \cmidrule{1-2}
			\textbackslash\%               & \%               \\ \cmidrule{1-2}
			\textbackslash\$               & \$               \\ \cmidrule{1-2}
			\textbackslash\#               & \#               \\ \cmidrule{1-2}
			\textbackslash\{               & \{               \\ \cmidrule{1-2}
			\textbackslash\}               & \}               \\ \cmidrule{1-2}
			\end{talltblr}}
	\end{column}

	\end{columns} 
\end{frame}                       
	\end{minted}
\end{code}

Wie zu sehen, wird in Beamer ebenfalls mit \texttt{sections} gearbeitet. Um einen neuen Slide zu erstellen wird 
\texttt{\textbackslash begin\{frame\}} genutzt. Die \texttt{columns} Umgebung ermöglicht das Unterteilen des Frames 
in mehrere Spalten, welche daraufhin als einzelne \texttt{column} definiert werden. Dabei kann die Textbreite angegeben werden.\\
Listings müssen über eine besondere Umgebung definiert und eingebunden werden, da Beamer ansonsten nicht mit \texttt{minted} umgehen kann. 

\begin{code}{Beamer Frame mit Listing}{beamer_frame_listing}
		\inputminted{latex}{src/beamerlisting.txt}
\end{code}