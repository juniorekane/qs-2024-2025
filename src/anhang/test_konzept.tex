\newpage
\subsection{Testkonzept}


Ein strukturiertes Testkonzept ist essenziell, um die Qualität und Stabilität der entwickelten Hausverwaltungssoftware sicherzustellen.  
Da es sich um einen Prototype handelt, fokussieren wir uns auf technische Tests zur Funktionsprüfung und verzichten auf umfassende Usability- oder Systemtests.  
Unser Ziel ist es, sicherzustellen, dass die Kernfunktionen fehlerfrei arbeiten, Daten korrekt verarbeitet werden und das System auch unter Last stabil bleibt.  
Die Tests orientieren sich an etablierten Softwaretestverfahren und wurden so konzipiert, dass sie eine möglichst hohe Abdeckung der Anforderungen gewährleisten.

\subsubsection{Testziele und Strategie}

\textbf{Testziele}

\begin{itemize}
	\item Sicherstellen, dass die Kernfunktionen korrekt arbeiten
	\item Prüfen, ob Module korrekt interagieren
	\item Fehlermeldungen und ungültige Eingaben testen
\end{itemize}

\textbf{Teststrategie}

\begin{itemize}
	\item Zuerst einzelne Komponenten testen (Unit-Tests)
	\item Danach prüfen, ob die Module zusammenarbeiten (Integrationstests)
	\item Überprüfung der Systemfunktionen (Funktionstests)
	\item Bewusst falsche Eingaben ausprobieren (Negative Tests)
\end{itemize}

\subsubsection{Ausgewählte Testverfahren und Begründung}

\begin{center}
	\begin{talltblr}[caption={Ausgewählte Testverfahren}, label={Testverfahren}]{width=0.9\textwidth, colspec={X[3,l,m] X[5,c,m]X[5,l,m]}}\toprule
		\textbf{Testverfahren} & \textbf{Einsatzbereich} & \SetCell[c=1]{c} \textbf{Begründung} \\ \midrule
		
		Unit-Tests & Einzelne Funktionen wie Datenvalidierung, ID-Format, Speicherung von Ablesewerten & Frühes Erkennen von Fehlern in einzelnen Modulen \\ \cmidrule{1-3}
		Funktionstests  & Überprüfung der gesamten Funktionalität wie Zählerverwaltung, Ablesungen, Filterung & Verifizierung der implementierten Anforderungen \\ \cmidrule{1-3}
		Performance-Tests  & Messung der Ladezeiten der Verbrauchsanzeige & Sicherstellen, dass das System auch mit vielen Gebäuden/Zählern performant bleibt \\ \cmidrule{1-3}
		Negative Tests  & Eingabe ungültiger Werte (z. B. leere Felder, falsche ID, negatives Datum) & Sicherstellen, dass das System Fehlersituationen richtig behandelt \\ \bottomrule

	\end{talltblr}
\end{center}

\subsubsection{Testumgebung und Testfälle}

\textbf{Testumgebung}
\begin{itemize}
	\item Der Prototyp wird in einer lokalen Entwicklungsumgebung getestet.
	\item Es wird eine Testdatenbank mit Dummy-Daten erstellt. Die Persistenz der Daten wird durch die Nutzung von json-Datei-Format gewährleistet, da es sich bei uns um eine Testumgebung für einen Prototyp.
\end{itemize}



Mit diesem Testkonzept wollen wir sicherstellen, dass die wichtigsten Funktionen des Prototyps getestet werden, ohne unnötig viel Zeit in realistische (wir meinen hier eine produktive Umgebung.)\\
oder nicht notwendige Tests zu investieren. Die Kombination aus Unit-Tests, Integrationstests, Funktionstests und Negative Tests reicht aus, um die Qualität und Stabilität des Prototyps sicherzustellen.
