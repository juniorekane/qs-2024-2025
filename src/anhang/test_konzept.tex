\newpage
\subsection{Testkonzept}


\newcommand{\konzept}[4]{%
    \subsection*{titel}
    \vspace{0.3cm}
}

\subsubsection{Einleitung}

Durch dieses Testkonzept haben wir versucht die grundlegenden Testverfahren zu beschhreiben, die zur Überprüfung der Krenfunktionalitäten unseres Prototyps verwendet werden.\\
Da es sich lediglich um eine kleines Projekt handelt, liegt der Fokus bei uns auf die technischen Tests zur Funktionsprüfung, anstatt systemweit deckende Funktionalitäten oder System- oder Usability-Tests.\\
Unser Ziel ist es, die wichtigsten Funktionen zu validieren, um eine fehlerfreie und konsistendte Prototyp-Umsetzung sicherzustellen.

\subsubsection{Testziele und Strategie}

\textbf{Testziele}

\begin{itemize}
	\item Sicherstellen, dass die Kernfunktionen korrekt arbeiten
	\item Prüfen, ob Module korrekt interagieren
	\item Fehlermeldungen und ungültige Eingaben testen
\end{itemize}

\textbf{Teststrategie}

\begin{itemize}
	\item Zuerst einzelne Komponenten testen (Unit-Tests)
	\item Danach prüfen, ob die Module zusammenarbeiten (Integrationstests)
	\item Überprüfung der Systemfunktionen (Funktionstests)
	\item Bewusst falsche Eingaben ausprobieren (Negative Tests)
\end{itemize}

\subsubsection{Ausgewählte Testverfahren und Begründung}

\begin{center}
	\begin{talltblr}[caption={Ausgewählte Testverfahren}, label={Testverfahren}]{width=0.9\textwidth, colspec={X[3,l,m] X[5,c,m]X[5,l,m]}}\toprule
		\textbf{Testverfahren} & \textbf{Einsatzbereich} & \SetCell[c=1]{c} \textbf{Begründung} \\ \midrule
		
		Unit-Tests & Einzelne Funktionen wie Datenvalidierung, ID-Format, Speicherung von Ablesewerten & Frühes Erkennen von Fehlern in einzelnen Modulen \\ \cmidrule{1-3}
		Integrationstests  & Zusammenspiel der Module, z. B. Verknüpfung von Zähler, Wohnung und Gebäude & Sicherstellen, dass die Module korrekt miteinander arbeiten \\ \cmidrule{1-3}
		Funktionstests  & Überprüfung der gesamten Funktionalität wie Zählerverwaltung, Ablesungen, Filterung & Verifizierung der implementierten Anforderungen \\ \cmidrule{1-3}
		Negative Tests  & Eingabe ungültiger Werte (z. B. leere Felder, falsche ID, negatives Datum) & Sicherstellen, dass das System Fehlersituationen richtig behandelt \\ \bottomrule

	\end{talltblr}
\end{center}

\subsubsection{Testumgebung und Testfälle}

\textbf{Testumgebung}
\begin{itemize}
	\item Der Prototyp wird in einer lokalen Entwicklungsumgebung getestet.
	\item Es wird eine Testdatenbank mit Dummy-Daten erstellt.
\end{itemize}

\\
\textbf{Wichtige Testfälle}

\begin{center}
	\begin{talltblr}[caption={relevante Testfälle}, label={Testfälle}]{colspec={X[5,c,m]X[5,l,m]}, width=0.9\textwidth}\toprule
		Testfall & Erwartetes Ergebnis \\ \midrule
		Zähler-ID existiert nicht & Fehlermeldung: „Die eingegebene ID existiert nicht.“ \\ \cmidrule{1-2}
		Ablesewert ist negativ (-10) & Fehlermeldung: „Ungültiger Ablesewert.“ \\ \cmidrule{1-2}
		Eingabe eines zu langen Zähler-Codes & Fehlermeldung: „Zähler-ID muss 10 Zeichen haben.“ \\ \cmidrule{1-2}
		Korrekte ID eingeben & Zähler wird erfolgreich gefunden \\ \cmidrule{1-2}
		Eingabe einer gültigen Ablesung & Wert wird korrekt gespeichert \\ \bottomrule
	\end{talltblr}
\end{center}

\subsubsection{Fazit}

Mit diesem Testkonzept wollen wir sicherstellen, dass die wichtigsten Funktionen des Prototyps getestet werden, ohne unnötig viel Zeit in realistische (wir meinen hier eine produktive Umgebung.)\\
oder nicht notwendige Tests zu investieren. Die Kombination aus Unit-Tests, Integrationstests, Funktionstests und Negative Tests reicht aus, um die Qualität und Stabilität des Prototyps sicherzustellen.
