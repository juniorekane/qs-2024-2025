\newpage
\subsection{Konkrete Testfälle für die Hausverwaltungssoftware}

Die folgenden Testfälle überprüfen die wichtigsten Funktionen des Prototyps. Dabei werden \textbf{Unit-Tests, Integrationstests, Funktionstests und Negative Tests} berücksichtigt.

\footnotesize
\begin{center}
	\begin{talltblr}[caption={Testfälle für die Hausverwaltungssoftware}, label={tab:testcases}]{width=0.9\textwidth, colspec={X[1,l,m] X[3,c,m] X[3,l,m] X[3,l,m] X[3,l,m]}}\toprule

        \textbf{Test-ID} & \textbf{Beschreibung} & \textbf{Eingabe} & \textbf{Erwartetes Ergebnis} & \textbf{Testtyp} \\ \midrule
        TC-001 & Zähler-ID existiert nicht & `999-9999-9999` & Fehlermeldung: „Die eingegebene ID existiert nicht.“ & Negative Test \\ \cmidrule{1-5}
        TC-002 & Negativer Ablesewert & `-10` & Fehlermeldung: „Ungueltiger Ablesewert“ & Negative Test \\ \cmidrule{1-5}
        TC-003 & Zählerlänge & `1-2024-4567823` & Fehlermeldung: „Zähler-ID muss genau 14 Zeichen haben!“ & Negative Test \\ \cmidrule{1-5}
        TC-004 & Ablesedatum rückdatiert & `2000-01-01` & Fehlermeldung: „Datum darf nicht in der Vergangenheit liegen!“ & Negative Test \\ \cmidrule{1-5}
        TC-005 & Ablesewert kleiner als vorheriger Wert & Neuer Wert: `50`, alter Wert: `100` & Fehlermeldung: „Neuer Wert muss größer sein als der vorherige.“ & Negative Test \\ \cmidrule{1-5}
        TC-006 & Gültige Zähler-ID über die Suchfunktion eingeben & `1-2025-5487` & Zählerdetails werden angezeigt & Funktionstest \\ \cmidrule{1-5}
        TC-007 & Korrekte Ablesung speichern & Alter Wert: 100, Neuer Wert: `250` & Wert wird korrekt gespeichert & Funktionstest \\ \cmidrule{1-5}
        TC-008 & Ablesedatum in der Zukunft & Datum: `01.01.2030` & Wert wird gespeichert & Funktionstest \\ \cmidrule{1-5}
        TC-009 & Standard-Ableser bei fehlender Eingabe & Ableser nicht eingetragen & Standardwert „Unbekannt“ wird gespeichert & Funktionstest \\ \cmidrule{1-5}
        TC-010 & Historische Verbrauchswerte anzeigen & Es wird die Schnittstelle für Verbauchshistorie mit der Gebäude-ID "1" abgerufen & Ablesungen sollen als Liste zurückgegeben werden oder als Grafik in der Weboberfläche & Funktionstest \\ \cmidrule{1-5}
        TC-011 & Suchfunktion mit Teilstring & Eingabe: `123` & Zeigt alle Zähler mit `123` in der ID & Funktionstests \\ \cmidrule{1-5}
        TC-012 & Massive Abelsungen & 10000 Ablesungen & Es werden alle Ablesungen in maximal 60 Sekunden gespeichert und keine Daten gehen verloren & Perfomance-Tests \\ \cmidrule{1-5}
        TC-013 & Antwortzeit-Test & Es wird die index-Seite aufgerufen & Innerhalb von wenigen Millisekunden eine Antwort geliefert & Performance-Test \\ \cmidrule{1-5}
        TC-014 & Massive Zählererstellung & 10000 Strom-Zähler & Alle Zähler werden hinzugefügt ohne zu lange Wartezeit & Performance-Test \\ \cmidrule{1-5}
        TC-015 & Massive Gebäude erstellen & 10000 Strom-Zähler & Alle Gebäude werden hinzugefügt ohne zu lange Wartezeit & Performance-Test \\ \cmidrule{1-5}
        TC-016 & Datenspeicherung und Datenabruf im JSON-Format & Dummy-Gebäude-Daten & Gebäude-Daten sollen gespeichert und abgerufen werden können. & Unit-Test \\ \cmidrule{1-5}
        TC-017 & Zähler-ID-Generierung & 7 als Gebäude-ID und das aktuelle Jahr & Es soll eine gültige Zähler-ID generiert werden. & Unit-Test \\ \bottomrule
    \end{talltblr}
\end{center}

\normalsize