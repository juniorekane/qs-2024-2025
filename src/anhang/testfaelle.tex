\newpage
\subsection{Konkrete Testfälle für die Hausverwaltungssoftware}

Die folgenden Testfälle überprüfen die wichtigsten Funktionen des Prototyps. Dabei werden \textbf{Unit-Tests, Integrationstests, Funktionstests und Negative Tests} berücksichtigt.

\begin{center}
	\begin{talltblr}[caption={Testfälle für die Hausverwaltungssoftware}, label={tab:testcases}]{width=0.9\textwidth, colspec={X[1,l,m] X[3,c,m] X[3,l,m] X[3,l,m] X[3,l,m]}}\toprule

        \textbf{Test-ID} & \textbf{Beschreibung} & \textbf{Eingabe} & \textbf{Erwartetes Ergebnis} & \textbf{Testtyp} \\ \midrule
        TC-001 & Zähler-ID existiert nicht & `999-9999-9999` & Fehlermeldung: *„Die eingegebene ID existiert nicht.“* & Negative Test \\ \cmidrule{1-5}
        TC-002 & Gültige Zähler-ID eingeben & `123-2024-4567` & Zählerdetails werden angezeigt & Funktionstest \\ \cmidrule{1-5}
        TC-003 & Ablesewert negativ & `-10` als Ablesewert & Fehlermeldung: *„Ungültiger Ablesewert.“* & Negative Test \\ \cmidrule{1-5}
        TC-004 & Ablesewert kleiner als vorheriger Wert & Neuer Wert: `50`, alter Wert: `100` & Fehlermeldung: *„Neuer Wert muss größer sein als der vorherige.“* & Negative Test \\ \cmidrule{1-5}
        TC-005 & Korrekte Ablesung speichern & Neuer Wert: `250` & Wert wird korrekt gespeichert & Funktionstest \\ \cmidrule{1-5}
        TC-006 & Eingabe einer zu langen Zähler-ID & `123-2024-45678` (11 Zeichen) & Fehlermeldung: *„Zähler-ID muss genau 10 Zeichen haben.“* & Negative Test \\ 
        TC-007 & Filtern nach Gebäude und Zählertyp & Gebäude: `Haus A`, Zählertyp: `Strom` & Liste zeigt nur Stromzähler von `Haus A` & Integrationstest \\ 
        TC-008 & Ablesedatum in der Zukunft & Datum: `01.01.2030` & Wert wird gespeichert & Funktionstest \\ \cmidrule{1-5}
        TC-009 & Ablesedatum rückdatiert & Datum: `01.01.2000` & Fehlermeldung: *„Datum darf nicht in der Vergangenheit liegen.“* & Negative Test \\ \cmidrule{1-5} 
        TC-010 & Standard-Ableser bei fehlender Eingabe & Ableser nicht eingetragen & Standardwert „Unbekannt“ wird gespeichert & Funktionstest \\ \cmidrule{1-5}
        TC-011 & Historische Verbrauchswerte anzeigen & Monat: `Januar` & Diagramm zeigt Verbrauchswerte für Januar & Funktionstest \\ \cmidrule{1-5}
        TC-012 & Suchfunktion mit Teilstring & Eingabe: `123` & Zeigt alle Zähler mit `123` in der ID & Integrationstest \\ \bottomrule
        \bottomrule
    \end{talltblr}
\end{center}
