\newpage
\subsection{Verbesserte Anforderungen auf Review-Basis}

% Verbesserte und präzisierte Version der Anforderungen nach dem Review
% Falls Änderungen vorgenommen wurden, eine Gegenüberstellung (Vorher-Nachher-Vergleich)

Nach der Überarbeitung der ursprünglichen Anforderungen haben wir die finalen Anforderungen für die Hausverwaltung definiert.
Diese berücksichtigen die Ergebnisse des Reviews und wurden klarer formuliert, widerspruchsfrei gestaltet und um spezifische Validierungsregeln ergänzt.
Die neuen Anforderungen bilden die Basis für die Implementierung des Prototyps und stellen sicher,
dass alle relevanten Aspekte der Hausverwaltung praxisnah und technisch umsetzbar sind.
Hier haben wir eine Unterscheidung zwischen funktionalen und nicht funktionalen Anforderungen gemacht.

\footnotesize

\textbf{Funktionale Anforderungen}

\begin{center}
	\begin{talltblr}[caption={Funktionale Anforderungen}, label={neue funktionale Anforderungen}]{width=0.9\textwidth, colspec={X[1,l,m] X[5,c,m]X[5,l,m]}}\toprule
		\textbf{Nr.} & \textbf{Anforderung} & \SetCell[c=1]{c} \textbf{Beschreibung} \\ \midrule
		
		F1 & Gebäudestruktur verwalten & Gebäude können mehrere Eingänge haben, jede Wohnung hat eine eindeutige ID. \\ \cmidrule{1-3}
		F2  & Zählertypen verwalten & Unterstützte Typen: Strom, Gas, Wasser. Die Liste ist erweiterbar, indem neue Typen über eine Konfigurationsdatei hinzugefügt werden. \\ \cmidrule{1-3}
		F3  & Zählerverwaltung & Jeder Zähler hat eine eindeutige ID (Gebäude-Jahr-Nummer). Jeder Zähler gehört zu einer Wohnung und einem Zählertyp. Er speichert den letzten Ablesewert, das letzte Ablesedatum und die Ablesemethode. \\ \cmidrule{1-3}
		F4  & Datenfilterung & Filter nach Gebäude, Wohnung, Zählertyp und Zeitraum. \\ \cmidrule{1-3}
		F5  & Zählerablesung & Zählerwerte können nur mit aktuellem oder zukünftigen Datum erfasst werden. Korrekturen sind nur für Admins erlaubt. Negative Werte sind nicht zulässig. Falls der neue Wert kleiner als der vorherige ist, gibt es eine Fehlermeldung. \\ \cmidrule{1-3}
		F6  & Fehlermeldungen & Falls eine Zähler-ID nicht existiert, erscheint „Die eingegebene ID existiert nicht. Bitte überprüfen Sie Ihre Eingabe.“ Falls eine Wohnung keiner ID zugeordnet ist, erscheint „Dieser Zähler ist keiner Wohnung zugeordnet.“ \\ \cmidrule{1-3}
		F7  & Verbrauchsanzeige & Historische Verbrauchswerte sind für die letzten 12 Monate abrufbar. Eine grafische Darstellung ist möglich. \\ \cmidrule{1-3}
		F8  & Ableser-Informationen & Ableser können Hauswart, Mieter oder Energieversorger sein. Falls keine Information vorhanden ist, wird „Unbekannt“ eingetragen.\\ \cmidrule{1-3}
		F9  & Bearbeiten und  Löschen von Gebäuden & Gebäude können direkt bearbeitet oder gelöscht werden.\\ \cmidrule{1-3}
		F10  & Zurück-Buttons auf allen Seiten & Verbesserte Navigation in der Anwendung.\\ \cmidrule{1-3}
		F11  & Gebäude auswählen vor Verbrauchsanzeige & Nutzer müssen erst ein Gebäude wählen, bevor Verbrauchsdaten angezeigt werden.\\ \cmidrule{1-3}
		F12  & Direkte Weiterleitung bei nur einem Gebäude & Wenn nur ein Gebäude existiert, wird die Verbrauchsanzeige sofort geladen.\\ \cmidrule{1-3}
		F13  & Unterschiedliche Speicherung für aktuelle & historische Verbrauchsdaten: \verbrauch_aktuell_X.png und \verbrauch_historie_X_YYYY-MM-DD.png werden getrennt gespeichert. \\ \bottomrule

	\end{talltblr}
\end{center}

\newpage
\textbf{Nicht-funktionale Anforderungen}

\begin{center}
	\begin{talltblr}[caption={Nicht-Funktionale Anforderungen}, label={nicht funktionale Anforderungen}]{width=0.9\textwidth, colspec={X[1,l,m] X[5,c,m]X[5,l,m]}}\toprule
		\textbf{Nr.} & \textbf{Anforderung} & \SetCell[c=1]{c} \textbf{Beschreibung} \\ \midrule
		
		NF1 & Zeitraum für die Verbrauchsanzeige im Diagramm sichtbar & Das Diagramm zeigt den Zeitraum der Messung an (z. B. „März 2024 - Februar 2025“). \\ \cmidrule{1-3}
		NF2  & Letzte 12 Monate immer anzeigen (auch ohne Werte) & Die Verbrauchsanzeige berücksichtigt automatisch die letzten 12 Monate. Fehlende Werte werden als „0“ dargestellt. \\ \cmidrule{1-3}
		NF3  & Farbliche Kennzeichnung der Zähler in der Verbrauchsanzeige & Jeder Zähler erhält eine eindeutige Farbe zur besseren Unterscheidung. \\ \cmidrule{1-3}
		NF4  & Optimierung der Antwortzeiten & Das System soll Verbrauchsdaten in unter 2 Sekunden berechnen und anzeigen. \\ \cmidrule{1-3}
		NF5  & Datenintegrität und Konsistenz & Ablesewerte dürfen nicht rückwirkend geändert werden (außer durch Admins).\\ \cmidrule{1-3}
		NF6  & Speicherung von Verbrauchsdaten gemäß Datenschutzbestimmungen & Verbrauchsdaten dürfen nur von autorisierten Nutzern eingesehen werden. \\ \cmidrule{1-3}
		NF7  & System skalierbar für große Datenmengen & Unterstützung für mindestens 100 Gebäude und 5000 Zähler.\\ \bottomrule

	\end{talltblr}
\end{center}

\normalsize