\newpage
\subsection{Verbesserte Anforderungen auf Review-Basis}

% Verbesserte und präzisierte Version der Anforderungen nach dem Review
% Falls Änderungen vorgenommen wurden, eine Gegenüberstellung (Vorher-Nachher-Vergleich)

Nach der Überarbeitung der ursprünglichen Anforderungen haben wir die finalen Anforderungen für die Hausverwaltung definiert.
Diese berücksichtigen die Ergebnisse des Reviews und wurden klarer formuliert, widerspruchsfrei gestaltet und um spezifische Validierungsregeln ergänzt.
Die neuen Anforderungen bilden die Basis für die Implementierung des Prototyps und stellen sicher,
dass alle relevanten Aspekte der Hausverwaltung praxisnah und technisch umsetzbar sind.

\footnotesize

\begin{center}
	\begin{talltblr}[caption={Verbesserte Anforderungen}, label={neue Anforderungen}]{width=0.9\textwidth, colspec={X[1,l,m] X[1,c,m]X[5,l,m]}}\toprule
		\textbf{Nr.} & \textbf{Anforderung} & \SetCell[c=1]{c} \textbf{Beschreibung} \\ \midrule
		
		1 & Gebäudestruktur & Ein Gebäude kann mehrere Eingänge haben. Ein Eingang ist eine physische oder logische Einheit, die Zugang zu Wohnungen ermöglicht. Jede Wohnung hat eine eindeutige ID. \\ \cmidrule{1-3}
		2  & Zählertypen & Unterstützte Typen: Strom, Gas, Wasser. Die Liste ist erweiterbar, indem neue Typen über eine Konfigurationsdatei oder Admin-Oberfläche hinzugefügt werden. \\ \cmidrule{1-3}
		3  & Zählerverwaltung & Jeder Zähler hat eine eindeutige 10-stellige ID (Format: `ZZZ-YYYY-NNNN`, z. B. `123-2024-4567`). Jeder Zähler gehört zu einer Wohnung und einem Zählertyp. Er speichert den letzten Ablesewert, das letzte Ablesedatum und die Ablesemethode. \\ \cmidrule{1-3}
		4  & Datenfilterung & Zähler können nach Gebäude, Wohnung, Zählertyp und Zeitraum gefiltert werden. Die Suche unterstützt Teilstringsuche. \\ \cmidrule{1-3}
		5  & Zählerablesung & Zählerwerte können nur mit aktuellem oder zukünftigen Datum erfasst werden. Korrekturen sind nur für Admins erlaubt. Negative Werte sind nicht zulässig. Falls der neue Wert kleiner als der vorherige ist, gibt es eine Fehlermeldung. \\ \cmidrule{1-3}
		6  & Fehlermeldungen & Falls eine Zähler-ID nicht existiert, erscheint „Die eingegebene ID existiert nicht. Bitte überprüfen Sie Ihre Eingabe.“ Falls eine Wohnung keiner ID zugeordnet ist, erscheint „Dieser Zähler ist keiner Wohnung zugeordnet.“ \\ \cmidrule{1-3}
		7  & Verbrauchsanzeige & Historische Verbrauchswerte sind für die letzten 12 Monate abrufbar. Eine grafische Darstellung ist möglich. Monatliche Verbrauchswerte können exportiert werden. \\ \cmidrule{1-3}
		8  & Ableser-Informationen & Ableser können Hauswart, Mieter oder Energieversorger sein. Falls keine Information vorhanden ist, wird „Unbekannt“ eingetragen. \\ \bottomrule

	\end{talltblr}
\end{center}

\normalsize