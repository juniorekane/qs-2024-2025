\subsection{Vorher-Nachher-Vergleich}

Wir haben auch einen Vergleich der alten und neuen Anforderungen an der Hausverwaltung erstellt, um Unklarheiten zu beseitigen, \\
Redundanzen zu vermeiden und eine bessere Umsetzbarkeit im Prototyp zu gewährleisten. Nachfolgend haben wir versucht, es nachvollziehbar zu machen, weshalb die verschiedene Änderungen da sind, weil wir weniger Anforderungen als ursprünglich haben.


\newcommand{\anforderungVergleich}[4]{%
    \subsection*{Anforderung #1: #2}
    \textbf{Vorher:} #3
    
    \vspace{0.3cm}
    \textbf{Nachher:} #4

    \vspace{0.3cm}
}

\anforderungVergleich{1}{Gebäudestruktur (1..n Gebäude, Eingänge, Wohnungen, Zähler)}
{Die Gebäudestruktur enthält mehrere Gebäude, Eingänge und Wohnungen. Es war unklar, ob ein Eingang eine physische oder logische Einheit ist.}
{Ein Gebäude kann mehrere Eingänge haben. Ein Eingang ist eine physische oder logische Einheit, die Zugang zu Wohnungen ermöglicht. Jede Wohnung hat eine eindeutige ID.}

\textbf{Begründung:}  
Die Definition des „Eingangs“ wurde ergänzt, um Unklarheiten zu beseitigen.

\anforderungVergleich{2}{Zählertypen (Strom, Gas, Wasser)}
{Nur Strom, Gas und Wasser sind als Zählertypen definiert. Es war unklar, ob weitere Zählertypen ergänzt werden können.}
{Zählertypen sind erweiterbar und können über eine Konfigurationsdatei oder Admin-Oberfläche hinzugefügt werden.}

\textbf{Begründung:}  
Die Anforderung wurde verbessert, um zukünftige Erweiterungen zu ermöglichen.

\anforderungVergleich{3}{Zähler eindeutig identifizierbar (Zählernummer)}
{Die Zählernummer soll eindeutig sein, aber es gibt keine Spezifikation zum Format oder zur Länge.}
{Zähler haben eine 10-stellige eindeutige ID (Format: `ZZZ-YYYY-NNNN`, z. B. `123-2024-4567`). Jeder Zähler gehört zu einer Wohnung und einem Zählertyp.}

\textbf{Begründung:}  
Das Format der ID wurde spezifiziert, um die Implementierung zu erleichtern.

\anforderungVergleich{4}{Auswahl von Daten per Selektion in der Struktur}
{Benutzer sollen Daten über eine Selektion in der Struktur auswählen können. Es gibt keine genaue Beschreibung, welche Filter oder Suchoptionen existieren.}
{Benutzer können Zähler nach Gebäude, Wohnung, Zählertyp und Zeitraum filtern. Die Suchfunktion unterstützt Teilstringsuche.}

\textbf{Begründung:}  
Klarstellung der Filter- und Suchmöglichkeiten, um die Funktion für den Prototyp konkret umzusetzen.

\anforderungVergleich{5}{Zähler haben einen Ablesewert (ganze Zahl)}
{Die Ablesewerte sind Ganzzahlen, aber es war nicht definiert, ob falsche Eingaben möglich sind oder korrigiert werden können.}
{Zählerwerte müssen aktuell oder in der Zukunft liegen. Negative Werte sind nicht zulässig. Admins können Werte korrigieren.}

\textbf{Begründung:}  
Ergänzung von Validierungsregeln zur Fehlervermeidung.

\anforderungVergleich{6}{Zähler sind über ihre ID zu finden}
{Es war unklar, was passiert, wenn eine eingegebene ID nicht existiert.}
{Falls eine ID nicht existiert, wird eine Fehlermeldung angezeigt: *„Die eingegebene ID existiert nicht. Bitte überprüfen Sie Ihre Eingabe.“*}

\textbf{Begründung:}  
Explizite Fehlermeldungen wurden definiert, um das Systemverhalten vorhersehbar zu machen.

\anforderungVergleich{7}{Zähler sollen abgelesen werden (Eingabe von Datum und Wert)}
{Es gab keine Validierung für vergangene oder zukünftige Ablesewerte.}
{Ablesewerte werden nur akzeptiert, wenn sie größer als der vorherige Wert sind. Fehlermeldung bei ungültigen Eingaben.}

\textbf{Begründung:}  
Logische Konsistenzregeln wurden ergänzt, um Fehler zu vermeiden.

\anforderungVergleich{8}{Zähler und Datum laufen nur vorwärts}
{Es war unklar, ob rückdatierte Werte erlaubt sind oder ob es Grenzwerte für Eingaben gibt.}
{Diese Anforderung wurde mit „Zählerablesung“ zusammengeführt, da sie dieselbe Validierung betrifft.}

\textbf{Begründung:}  
Diese Anforderung ist jetzt Teil von „Zählerablesung“, da eine rückdatierte Eingabe bereits als Fehler behandelt wird.

\anforderungVergleich{9}{Weitere Ableseinformationen eingeben (Ablesung, Schätzung)}
{Es war nicht spezifiziert, welche Ablesemethoden (manuell, automatisch, Schätzung) hinterlegt werden können.}
{Ablesemethode (manuell, automatisch, Schätzung) wird gespeichert.}

\textbf{Begründung:}  
Diese Information gehört zur Ablesung und wurde integriert.

\anforderungVergleich{10}{Ableserinformationen eingeben (Hauswart, Mieter, Energieversorger)}
{Es war nicht klar, ob mehrere Ableser für einen Zähler existieren können oder ob eine Standardzuweisung existiert.}
{Ableser können Hauswart, Mieter oder Energieversorger sein. Falls keine Information vorhanden ist, wird „Unbekannt“ eingetragen.}

\textbf{Begründung:}  
Ein Standardwert wurde definiert, um unvollständige Daten zu vermeiden.

\anforderungVergleich{11}{Verbrauch berechnen und Anzeigen}
{Es war unklar, ob historische Verbrauchswerte gespeichert oder angezeigt werden können.}
{Historische Verbrauchswerte der letzten 12 Monate sind abrufbar. Grafische Darstellung und Export möglich.}

\textbf{Begründung:}  
Die Anforderungen an die Verbrauchsanzeige wurden konkretisiert und die Möglichkeit eines Exports ergänzt.


Mit dieser Überarbeitung haben wir die Anforderungen an die Hausverwaltung strukturiert, verständlich und technisch realisierbar formuliert.
Die neue Version stellt sicher, dass alle wichtigen Funktionalitäten klar definiert sind, notwendige Validierungsregeln integriert wurden und das System sowohl für Nutzer als auch für\\
Administratoren effizient funktioniert.
Durch die Analyse der ursprünglichen Anforderungen und die darauf basierenden Verbesserungen wurde eine solide Grundlage für die Entwicklung des Prototyps geschaffen.
Diese Anforderungen dienen als Leitfaden für die Implementierung und ermöglichen eine zielgerichtete und fehlerfreie Umsetzung.