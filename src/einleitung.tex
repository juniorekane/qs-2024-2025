\section{Einleitung}\label{Einleitung}

% Vorstellung des Projekts: Entwicklung einer Hausverwaltung
% Zielsetzung: Verbesserung der Qualität durch strukturierte Anforderungsanalyse und Testkonzept
% Relevanz des Themas im Kontext des Qualitätsmanagements und Software-Testings
% Überblick über die Inhalte der Arbeit

Die Verwaltung von Gebäuden und deren Energieverbrauch stellt in der Praxis eine zentrale Herausforderung dar.
Insbesondere in Mehrfamilienhäusern oder Wohnanlagen ist eine effiziente und übersichtliche Erfassung von Zählerständen erforderlich, um Verbrauchsdaten transparent zu machen und eine gerechte Abrechnung zu ermöglichen.
Im Rahmen dieses Projekts entwickeln wir als drei Studierende einen Prototyp für eine Hausverwaltungssoftware, die sich auf die digitale Erfassung, Verwaltung und Analyse von Zählerständen konzentriert.\par

Die Umsetzung erfolgt als webbasierte Anwendung, wobei der Fokus auf einer intuitiven Benutzeroberfläche und einer zuverlässigen Datenverarbeitung liegt.
Die Hausverwaltung ermöglicht es, Gebäude, Zähler und Verbrauchsdaten zu verwalten, Zählerablesungen zu dokumentieren und historische Verbrauchswerte grafisch darzustellen.
Dabei werden sowohl technische als auch organisatorische Aspekte berücksichtigt, um eine realitätsnahe und funktionale Lösung zu entwickeln.\par

Ein wesentlicher Bestandteil des Projekts ist das Review der Anforderungen sowie die Entwicklung eines fundierten Testkonzepts, um sicherzustellen, dass der Prototyp stabil, fehlerresistent und effizient arbeitet.
Im Rahmen unserer Ausarbeitung dokumentieren wir die einzelnen Projektschritte detailliert und analysieren die gewonnenen Erkenntnisse.
Unser Ziel ist es, ein möglichst praxisnahes und gut strukturiertes System zu entwerfen, das die wesentlichen Funktionen einer Hausverwaltung abbildet.\par

Die Entwicklung des Prototyps folgt einem iterativen Ansatz, bei dem wir zunächst die Anforderungen überprüft und überarbeitet haben, um Widersprüche oder Unklarheiten zu beseitigen.
Anschließend wurden konkrete Testfälle definiert, um die Kernfunktionen zu validieren. Die Tests umfassen funktionale Prüfungen, negative Tests sowie Leistungstests, um sowohl korrekte Funktionalität als auch Systemgrenzen zu ermitteln.
Schließlich wurde der Prototyp entsprechend der definierten Anforderungen und Testfälle umgesetzt.\par