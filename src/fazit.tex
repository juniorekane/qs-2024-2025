\section{Fazit}\label{sec:fazit}

% Vorstellung des Projekts: Entwicklung einer Hausverwaltung
% Zielsetzung: Verbesserung der Qualität durch strukturierte Anforderungsanalyse und Testkonzept
% Relevanz des Themas im Kontext des Qualitätsmanagements und Software-Testings
% Überblick über die Inhalte der Arbeit

Im Rahmen dieses Projekts haben wir eine Hausverwaltungssoftware prototypisch entwickelt und dabei einen umfassenden softwaretechnischen Prozess durchlaufen – von der Anforderungsanalyse über die Implementierung bis hin zur Qualitätssicherung.
Unser Ziel war es, ein System zu entwerfen, das die Erfassung, Verwaltung und Analyse von Verbrauchsdaten effizient unterstützt und dabei stabil, fehlerresistent und benutzerfreundlich ist.

Die Anforderungsanalyse bildete die Grundlage für die Entwicklung unseres Prototyps.
Durch ein technisches Review nach ISO 20246 konnten wir potenzielle Unklarheiten in den ursprünglichen Anforderungen identifizieren und gezielt verbessern.
Die Trennung zwischen funktionalen und nicht-funktionalen Anforderungen hat sich als vorteilhaft erwiesen, da sie die Strukturierung und Testbarkeit unseres Systems erleichtert hat.

Auf Basis dieser überarbeiteten Anforderungen wurde ein strukturiertes Testkonzept entwickelt.
Durch die Kombination von Unit-Tests, Funktionstests, negativen Tests und Performance-Tests konnten wir die wichtigsten Systemfunktionen validieren und sicherstellen, dass das System auch unter Last stabil bleibt.
Die durchgeführten Tests bestätigten die Korrektheit der Kernfunktionen und deckten potenzielle Optimierungsansätze auf, insbesondere in Bezug auf die Skalierbarkeit und Datenspeicherung.

Die Implementierung unseres Prototyps folgte einer modularen Architektur mit einem Flask-Backend, einer JSON-basierten Datenspeicherung und eine webbasierte Benutzeroberfläche.
Diese Architektur ermöglichte eine schnelle Entwicklung und eine einfache Testbarkeit.
Herausforderungen, wie die Validierung von Ablesedaten oder die Simulation von Lasttests, wurden durch gezielte Anpassungen und Optimierungen erfolgreich gemeistert.

Unsere Qualitätssicherungsmaßnahmen haben sich als effektiv erwiesen, um Fehler frühzeitig zu identifizieren und das System iterativ zu verbessern. 
Die Rückverfolgbarkeit zwischen Anforderungen und Testfällen trug wesentlich dazu bei, dass keine kritische Funktion ungetestet blieb.
Zudem zeigten unsere Performance-Tests, dass das System mit hohen Datenmengen umgehen kann, auch wenn eine relationale Datenbank für eine produktive Umgebung empfehlenswert wäre.

Zusammenfassend hat unser Prototyp die wesentlichen Anforderungen erfolgreich umgesetzt und bietet eine solide Grundlage für eine Weiterentwicklung.
Falls das System für eine produktive Umgebung ausgebaut werden soll, wären folgende Maßnahmen sinnvoll:

\begin{itemize} 
    \item Migration von JSON zu einer relationalen Datenbank für bessere Skalierbarkeit und Sicherheit. 
    \item Erweiterung der Performance-Tests mit größeren Datenmengen und realistischen Nutzungsszenarien. 
    \item Einführung automatisierter UI-Tests zur Überprüfung der Benutzerfreundlichkeit. 
    \item Implementierung weiterer Fehlerbehandlungen und Sicherheitsmechanismen. 
\end{itemize}

Dieses Projekt hat uns gezeigt, wie entscheidend eine strukturierte Anforderungsanalyse, ein fundiertes Testkonzept und iterative Qualitätssicherungsmaßnahmen für die erfolgreiche Entwicklung einer Software sind.
Die gewonnenen Erkenntnisse und Erfahrungen aus diesem Projekt werden uns in zukünftigen Softwareentwicklungsprojekten von großem Nutzen sein.