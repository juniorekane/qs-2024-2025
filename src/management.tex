\section{Qualitätsmanagement-Methoden in der Softwareentwicklung}\label{Qualitätsmanagement-Methoden in der Softwareentwicklung}

Aufgrund  der zentralen Rolle, was Qualitätssicherung in der Softwareentwicklung spielt, sollten wir uns diese Gelegenheit nicht entgehen lassen, eine reflektierende und wissenschaftliche Betrachtung der angewendeten QS-Methoden in unserem Projekts zu liefern, daher sollten wir auch\\
in diesem Zusammenhang die Bedeutung von Qualitätssicherung herausarbeiten und dann konkret auf die Anwendung von QS-Methoden in unserem Projekt eingehen.\par

\subsection{Relevanz der Qualitätssicherung}

In der Softwareentwicklung spielt die Qualitätssicherung (QS) eine entscheidende Rolle, um sicherzustellen, dass Anwendungen fehlerfrei funktionieren, effizient arbeiten und benutzerfreundlich sind.\par

Mangelhafte QS kann zu Fehlfunktionen, Sicherheitslücken und höheren Kosten führen, da nachträgliche Fehlerkorrekturen aufwendiger sind als eine frühzeitige Fehlervermeidung.\par

Gerade in unserem Projekt, einem hausverlatungssystem, ist eine präzise Qualitätssicherung unerlässlich. Die Anwendung muss zuverlässig arbeiten, um sicherzustellen, dass Verbrauchsdaten korrekt verarbeitet unnd angezeigt werden,\\
damit eine fehlerfreie Abrechnung möglich ist. Außerdem müssen Fehlermeldung konsisten und verständlich sein, damit Nutzer wissen, was schiefgelaufen ist und wie sie das Problem beheben können.
Zudem soll die Anwendung auch bei hoher Nutzerlast stabil bleiben, insbesondere wenn viele Gebäude und Zähler verwaltet werden müssen.
Um diese Risiken zu minimieren, haben wir gezielt QS-Maßnahmen in unser Projekt integriert.\par

Unsere wichtigsten Maßnahmen zur Sicherstellung der Softwarequalität sind:

\begin{itemize}
    \item Ein Anforderungs-Review nach \textbf{ISO 20246}, um Unklarheiten frühzeitig zu identifizieren und widersprüchliche Anforderungen zu vermeiden.
    \item Ein strukturiertes Testkonzept, das verschiedene Testarten wie Unit-Tests, Funktionstests, Negative Tests und Performance-Tests umfasst.
    \item Traceability zwischen Anforderungen und Testfällen, um sicherzustellen, dass jede relevante Anforderungen auch tatsächlich getestet wird.
\end{itemize}


\subsection{Anwendung von QS-Methoden im Projekt}

Die Qualitätssicherung unseres Projekts erfolgte in mehreren Phasen, um eine hohe Softwarequalität und Stabilität sicherzustellen. 
Dabei haben wir gezielt Maßnahmen ergriffen, die Fehler frühzeitig identifizieren und die Nachverfolgbarkeit der Anforderungen verbessern.\par

Ein zentraler Bestandteil unserer Qualitätssicherung war die direkte Verknüpfung der Anforderungen mit den Testfällen.
Dadurch konnten wir sicherstellen, dass jede implementierte Funktion einer definierten Anforderung entspricht und getestet wird.
Alle Testfälle wurden aus den überarbeiteten Anforderungen abgeleitet und Unser technisches Review half uns, Anforderungen von der Implementierung zu präzisieren und spätere Fehler zu vermeiden.\par

Anforderungen sind ein wichtiger Punkt als Basis für die Implementierung vom Prototyp, dennoch sollten wir auch nicht vergessen, dass es noch wichtiger ist, die Prozesse zu dokumentieren, besonders die Tests. 
Eine strukturierte Testdokumentation war essenziell, um die Testergebnisse systematisch zu erfassen und nachvollziehbar zu machen.
Die Testergebnisse haben wir in einer Tabelle festgehalten, sodass jede Test-ID direkt mit einer Anforderung verknüpft ist. Außerdem haben wir fehlerhafte Test genau analysieret und dies führte zu gezielten Verbesserungen im Code.
Durch diese automatisierte Nachverfolgbarkeit konnten wir effizient Schwachstellen identifizieren und beheben, wodurch sich die Qualität des Prototyps erheblich verbesserte.\par

\subsection{Lessons Learned}

Wir können es nicht in Abrede stellen, dass uns die Anwendung von Qualitätssicherungsmethoden wertvolle Erkenntnisse gebracht hat, die über den reinen Testprozess hinausgehen.
Wie wir diese Qualitätsmanßnahmen konkrekt in unserem Projekt umgesetzt haben, wird im Folgenden erläutert:

\begin{itemize}
    \item \textbf{Früheres Review der Anforderungen hilft, Fehler zu vermeiden:} Unklare oder fehlerhafte Anforderungen führen zu einem erhöhten Testaufwand und Nachbesserungen.
    Durch unser strukturiertes Anforderungs-Review konnten wir dies minimieren und von Anfang an präzisere Anforderungen definieren.
    \item \textbf{automatisierte Tests sparen Zeit und erhöhen die Testabdeckung:} Manuelle Tests sind zeitaufwendig und fehleranfällig beispielsweise durch Konfigurationsfehler oder Eingabefehler.
    Mit \texttt{pytext} konnten wir wiederholbare Tests automatisieren und schneller fehler identifizieren.
    \item \textbf{Performance-Tests sind essenziell für die Skarlierbarkeit:} Unsere Lasttests zeigten, dass das System stabil mit großen Datenmengen umgehen kann.
    Allerdings haben wir erkannt, dass für eine echte Skalierbarkeit eine relationale Datenbank vorteilhafter wäre als die aktuelle JSON-basierte Speicherung.
    \item \textbf{Fehlermeldungen sind genauso wichtig wie korrekte Funktionen:} Nicht nur fehlerfreie Funktionen, sondern auch verständliche Fehlermeldungen tragen zur Benutzerfreundlichkeit bei.
    Durch gezielte Tests konnten wir sicherstellen, dass Fehlermeldungen für ungültige Zähler-IDs, falsche Ablesewerte oder negative Werte klar und nachvollziehbar sind.
\end{itemize}
