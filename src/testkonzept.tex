\newpage
\section{Testkonzept}\label{sec:testkonzept}

Ein strukturiertes Testkonzept ist essenziell, um die Qualität und Stabilität der entwickelten Hausverwaltungssoftware sicherzustellen.  
Da es sich um einen Prototype handelt, fokussieren wir uns auf technische Tests zur Funktionsprüfung und verzichten auf umfassende Usability- oder Systemtests.  
Unser Ziel ist es, sicherzustellen, dass die Kernfunktionen fehlerfrei funktionieren, Daten korrekt verarbeitet werden und das System auch unter Last stabil bleibt.
Die Tests orientieren sich an etablierten Softwaretestverfahren und wurden so konzipiert, dass sie eine möglichst hohe Abdeckung der Anforderungen gewährleisten.

\subsection{Testziele und Strategie}\label{subsec:testziele-und-strategie}

Für unser Testkonzept haben wir uns klare Testziele definiert, damit wir den Fokus behalten können und nicht unnötig viele Szenarien testen müssen.
Unsere Hauptziele sind:
\begin{itemize}
	\item Sicherstellen, dass die Kernfunktionen der Software korrekt arbeiten.
	\item Überprüfung, ob Module und Komponenten korrekt zusammenarbeiten.
	\item Validierung der Fehlerbehandlung durch gezielte Eingabe ungültiger Werte.
	\item Sicherstellung der Performance unter hoher Last.
\end{itemize}

Unser Testkonzept folgt einer auf \texttt{Botton-Up-Ansatz} basierten Strategie, bei der zunächst einzelne Komponente geprüft und anschließend System- und Performancetests durchgeführt werden:\par
\begin{itemize}
	\item \textbf{Zunächst Unit-Tests}: Isolierte Tests einzelner Funktionen (z.B. Validierung von Zähler-IDs oder Verbrauchsdaten).
	\item \textbf{Danach Funktionstests}: Prüfung der Geschäftslogik, u.a. Zählerverwaltung, Ablesungen und Verbrauchsberechnung.
	\item \textbf{Anschließend negative Tests}: Überprüfung der Fehlerbehandlung durch ungültige Eingaben.
	\item \textbf{Schließlich Performance-Tests}: Simulation von hoher Last, um die Skalierbarkeit zu überprüfen.
\end{itemize}

Wir wissen aus der Vorlesung, dass sich Softwaretests grundsätzlich in drei Kategorien unterteilen lassen:

\begin{itemize}
	\item \textbf{White-Box-Testing:} Interne Logik der Software wird geprüft, Code-Abdeckung ist entscheidend.
	\item \textbf{Black-Box-Testing:} Tests erfolgen ohne Kenntnis des Quellcodes, Fokus liegt auf den Ein- und Ausgaben des Systems.
	\item \textbf{Gray-Box-Testing:} Kombination aus White-Box und Black-Box, teilweise Kenntnisse über den Code werden verwendet.
\end{itemize}

Aus diesem Grund  haben wir uns eine Kombination aus \textbf{Black-Box-Testing} und \textbf{Gray-Box-Testing} entschieden, um die Funktionalität aus Nutzersicht zu prüfen und gezielt Fehlerfälle im Code zu analysieren.\\
Wir setzen auch die \textbf{Äquivalenzklassenbildung} ein, um Eingaben in Gruppen zu testen und mit wenigen Tests eine hohe Abdeckung zu erreichen.
Enorm hat uns auch \textbf{Boundary-Value-Testing} geholfen, Grenzwerte (z.B. minimale und maximale Verbrauchswerte) gezielt zu überprüfen.

Unsere Entscheidung, uns auf diese Techniken zu konzentrieren, basiert auf der Tatsache, dass unser
Prototyp eine Webanwendung ist, wobei der Fokus auf der Funktionalität der Schnittstellen und Datenverarbeitung liegt.
Die Äquivalenzklassenbildung reduziert die Anzahl der Testfälle, während trotzdem eine breite Abdeckung erreicht wird.
Letztendlich ist Boundary-value-Testing essenziell für die Überprüfung von Verbrauchswerten und Ablesedaten.

\subsection{Testarten}\label{subsec:testarten}

Um unser System effizient zu testen, haben wir vier Teststufen definiert:

\footnotesize
\begin{center}
	\begin{talltblr}[caption={Testarten}, label={Testarten}]{width=0.9\textwidth, colspec={X[3,l,m] X[5,c,m]X[5,l,m]}}\toprule
		\textbf{Testverfahren} & \textbf{Ziel} & \SetCell[c=1]{c} \textbf{Begründung} \\ \midrule
		
		Unit-Tests & Einzelne Funktionen wie Datenvalidierung, ID-Format, Speicherung von Ablesewerten & Frühes Erkennen von Fehlern in einzelnen Modulen. \\ \cmidrule{1-3}
		Funktionstests  & Überprüfung der gesamten Funktionalität wie Zählerverwaltung, Ablesungen, Filterung, Verbrauchsanzeige & Sicherstellung der korrekten Umsetzung der Anforderungen. \\ \cmidrule{1-3}
		Performance-Tests  & Simulation hoher Last durch 1000+ gleichzeitige Ablesungen & Sicherstellung, dass das System auch mit vielen Gebäuden und Zählern performant bleibt. \\ \cmidrule{1-3}
		Negative Tests  & Eingabe ungültiger Werte (z.B. leere Felder, falsche ID, negatives Datum) & Überprüfung der Fehlerbehandlung und Robustheit des Systems. \\ \bottomrule

	\end{talltblr}
\end{center}
\normalsize

\subsection{Testumgebung und Testdaten}\label{subsec:testumgebung-und-testdaten}

\subsubsection{Testumgebung}

Unsere Testumgebung haben wir versucht, so einfach wie möglich zu halten, damit wir nicht den Rahmen überspringen.
Hier sind die Kernpunkte unserer Testumgebung aufgelistet:
\begin{itemize}
	\item Der Prototyp wird in einer lokalen Entwicklungsumgebung als Flask-Anwendung entwickelt und getestet.
	\item Die Tests werden mithilfe von pytest automatisiert durchgeführt.
    \item Performance-Tests erfolgen durch Simulation hoher Anfragen über eine Flask-Testumgebung.
	\item Erstellung von Testfällen erfolgt nach den Prinzipien von Äquivalenzklassenbildung und Grenzwertanalyse.
\end{itemize}

\subsubsection{Testdaten}

Zur Testen gehören auch Testdaten zur Simulation, weil wir noch bei einem Prototyp sind, dessen Einsatz in einer produktiven Umgebung geplant ist.

\begin{itemize}
	\item Eine Testdatenbank mit Dummy-Daten wird verwendet.
	\item Persistenz der Daten erfolgt über JSON-Dateien.
	\item Für Lasttests werden 1000 simultane Ablesungen simuliert.
    \item Testfälle für Grenzwerte und ungültige Werte (z.B. negative Ablesungen) wurden vorbereitet.
\end{itemize}

\subsection{Ableitung konkreter Testfällen}\label{subsec:ableitung-konkreter-testfallen}

Die Qualität eines Softwareprojekts steht sowie es auch fällt mit der sorgfältigen Überprüfung seiner Funktionen.
Daher haben wir aus unseren Anforderungen gezielt abgeleitet, um sicherzustellen, dass der Prototyp alle Anforderungen erfüllt und fehlerfrei arbeitet.
Wir haben wie oben schon aufgelistet, vier Testarten gewählt

\newpage
\footnotesize
\begin{center}
	\begin{longtblr}[caption={Testfälle für die Hausverwaltungssoftware}, label={tab:testcases}]{width=0.9\textwidth, colspec={X[1,l,m] X[3,c,m] X[3,l,m] X[3,l,m] X[3,l,m]}}
		\toprule
        \textbf{Test-ID} & \textbf{Beschreibung} & \textbf{Eingabe} & \textbf{Erwartetes Ergebnis} & \textbf{Testtyp} \\ \midrule
        TC-001 & Zähler-ID existiert nicht & `999-9999-9999` & Fehlermeldung: „Die eingegebene ID existiert nicht.“ & Negative Test \\ \cmidrule{1-5}
        TC-002 & Negativer Ablesewert & `-10` & Fehlermeldung: „Ungültiger Ablesewert“ & Negative Test \\ \cmidrule{1-5}
        TC-003 & Zählerlänge & `1-2024-4567823` & Fehlermeldung: „Zähler-ID muss genau 14 Zeichen haben!“ & Negative Test \\ \cmidrule{1-5}
        TC-004 & Ablesedatum rückdatiert & `2000-01-01` & Fehlermeldung: „Datum darf nicht in der Vergangenheit liegen!“ & Negative Test \\ \cmidrule{1-5}
        TC-005 & Ablesewert kleiner als vorheriger Wert & Neuer Wert: `50`, alter Wert: `100` & Fehlermeldung: „Neuer Wert muss größer sein als der vorherige.“ & Negative Test \\ \cmidrule{1-5}
        TC-006 & Gültige Zähler-ID über die Suchfunktion eingeben & `1-2025-5487` & Zählerdetails werden angezeigt & Funktionstest \\ \cmidrule{1-5}
        TC-007 & Korrekte Ablesung speichern & Alter Wert: 100, Neuer Wert: `250` & Wert wird korrekt gespeichert & Funktionstest \\ \cmidrule{1-5}
        TC-008 & Ablesedatum in der Zukunft & Datum: `01.01.2030` & Wert wird gespeichert & Funktionstest \\ \cmidrule{1-5}
        TC-009 & Standard-Ableser bei fehlender Eingabe & Ableser nicht eingetragen & Standardwert „Unbekannt“ wird gespeichert & Funktionstest \\ \cmidrule{1-5}
        TC-010 & Historische Verbrauchswerte anzeigen & Es wird die Schnittstelle für Historiographer mit der Gebäude-ID \(`1'\) abgerufen & Ablesungen sollen als Liste zurückgegeben werden oder als Grafik in der Weboberfläche & Funktionstest \\ \cmidrule{1-5}
        TC-011 & Suchfunktion mit Teilstring & Eingabe: `123` & Zeigt alle Zähler mit `123` in der ID & Funktionstests \\ \cmidrule{1-5}
        TC-012 & Massive Ablesungen & 10000 Ablesungen & Es werden alle Ablesungen in maximal 60 Sekunden gespeichert und keine Daten gehen verloren & Performance-Tests \\ \cmidrule{1-5}
        TC-013 & Antwortzeit-Test & Es wird die index-Seite aufgerufen & Innerhalb von wenigen Millisekunden eine Antwort geliefert & Performance-Test \\ \cmidrule{1-5}
        TC-014 & Massive Zählererstellung & 10000 Strom-Zähler & Alle Zähler werden hinzugefügt ohne zu lange Wartezeit & Performance-Test \\ \cmidrule{1-5}
        TC-015 & Massive Gebäude erstellen & 10000 Gebäude & Alle Gebäude werden hinzugefügt ohne zu lange Wartezeit & Performance-Test \\ \cmidrule{1-5}
        TC-016 & Datenspeicherung und Datenabruf im JSON-Format & Dummy-Gebäude-Daten & Gebäude-Daten sollen gespeichert und abgerufen werden können. & Unit-Test \\ \cmidrule{1-5}
        TC-017 & Zähler-ID-Generierung & 7 als Gebäude-ID und das aktuelle Jahr & Es soll eine gültige Zähler-ID generiert werden. & Unit-Test \\ \bottomrule
    \end{longtblr}
\end{center}
\normalsize

Unser Testkonzept ist also sehr umfassend, deckt die meisten möglichen Szenarien ab und stellt sicher, dass der Prototyp funktionsfähig, robust und performant ist.
Durch die gewählte Kombination aus unseren Testarten wollen wir eine praxisnahe Qualitätssicherung erreichen.
